\chapter{Successioni e serie di funzioni}


\section{Successioni di funzioni}


\begin{define}[Successione di funzioni]
  Sia $f_n: \mathfrak{I} \to \mathbb{R}$ una funzione reale, con $n \in \mathbb{N}$, la successione 
  $\{f_n\}$ si dice \textbf{successione di funzioni}.
\end{define}

\begin{notation}
  $\{f_n\}$, $f_n$, $\{f_n\}_{n \in \mathbb{N}}$
\end{notation}

\begin{obs}
  Fissato $x_0 \in \mathfrak{I}$, $f_n(x_0)$ è una successione numerica.
\end{obs}

\begin{define}[Convergenza puntuale]
  Sia $f_n$ una successione di funzioni e sia $f: \mathfrak{I} \to \mathbb{R}$, si dice che 
  $f_n$ \textbf{converge puntualmente} a $f$ se:

  $$
    \forall x_0 \in \mathfrak{I}~f_n(x_0) \to f(x_0)
  $$
\end{define}


\begin{es}[I]
  Sia $\mathfrak{I} = [0, 1]$ e sia $f_n$:

  $$
    f_n(x) = x^n
  $$

  $f_n$ converge puntualmente a:

  $$
    f(x) =  \begin{cases*}
              0~\text{ se }~x \in [0, 1) \\
              1~\text{ se }~x = 1
            \end{cases*}
  $$
\end{es}

\begin{es}[II]
  Sia $\mathfrak{I} = [0, 1]$ e sia $f_n$:

  $$
    f_n(x) = \dfrac{\sin{(nx)}}{\sqrt{n}}
  $$

  $$
    f_n(x) \to f(x) = 0
  $$

  $$
    f'_n(x) = \sqrt{n}\cos{(nx)}
  $$
  $$
    f'_n(x) \to +\infty
  $$
\end{es}

\begin{es}[III]
  Sia $\mathfrak{I} = [0, 1]$ e sia $f_n$ così definita:

  ***grafico***

  $$
    f_n(x) \to f(x) = 0
  $$

  $$
    \int_{0}^{1}f_n(x)dx = 1~\forall n \in \mathbb{N}_0
  $$
  $$
    \lim_{n \to \infty} \int_{0}^{1}f_n(x)dx = 1 \neq \int_{0}^{1}\lim_{n \to \infty} f(x)dx = 0
  $$
\end{es}


\begin{prop}
  Se $f_n$ è continua $\forall n$ $\centernot \implies$ $f$ è continua.
\end{prop}

\begin{prop}
  Se $f_n$ è derivabile $\forall n$ $\centernot \implies$ $f$ è derivabile.
\end{prop}

\begin{prop}
  Se $f_n \in R(\mathfrak{I})$ $\forall n$ $\centernot \implies$ $f \in R(\mathfrak{I})$.
\end{prop}

\begin{es}
  Sia $f_n$ così definita nell'intervallo $[0, 1]$:

  $$
    f_n(x) =  \begin{cases*}
                \dfrac{1}{x}~\text{ se }~x \geqslant \dfrac{1}{n} \\
                0~\text{ se }~x < \dfrac{1}{n} \\
              \end{cases*}
  $$

  $$
    f_n(x) \to f(x) = \dfrac{1}{x}
  $$

  ma $f$ non è limitata in $[0, 1]$, quindi non è integrabile.
\end{es}

\begin{prop}
  Se $f_n \in R(\mathfrak{I})$ $\forall n$ e $f \in R(\mathfrak{I})$ $\centernot \implies$ $\lim_{n \to \infty}\int_{\mathfrak{I}}f_n(x)dx = \int_{\mathfrak{I}}f(x)dx = \int_{\mathfrak{I}}\lim_{n \to \infty}f_n(x)dx$.
\end{prop}


\subsection{Convergenza uniforme}

\begin{define}[Convergenza uniforme]
  Sia $\{f_n\}$ una successione di funzioni e sia $f: \mathfrak{I} \to \mathbb{R}$, $f_n$ 
  \textbf{converge uniformemente} a $f$ se:

  $$
    \forall \varepsilon > 0~\exists n_0 \in \mathbb{N}~\colon~|f_n(x) - f(x)| < \varepsilon~\forall n \geqslant n_0~\forall x \in \mathfrak{I}
  $$
\end{define}

\begin{notation}
  $f_n \uniformto f$
\end{notation}

\begin{obs}
  La scelta di $n_0$ non dipende da $x$.
\end{obs}

\begin{obs}
  $$
    f_n \uniformto f~\iff~\lim_{n \to \infty}\sup_{x \in \mathfrak{I}}|f_n(x) - f(x)| = 0
  $$
\end{obs}


\begin{thm}[Convergenza uniforme e continuità]
  Sia $f_n: \mathfrak{I} \to \mathbb{R}$ una successioni di funzioni, se $f_n$ è continua $\forall n$ e 
  $f_n \uniformto f$ $\implies$ $f$ è continua.
\end{thm}

\begin{proof}
  Dato che $f_n$ converge uniformemente a $f$, si ha che, fissato $x_0 \in \mathfrak{I}$:

  $$
    \forall \varepsilon > 0~\exists n_0 \in \mathbb{N}~\colon~|f_n(x) - f(x)| < \varepsilon~\forall n \geqslant n_0~\forall x \in \mathfrak{I}
  $$

  dato che $f_n$ è continua:

  $$
    \forall \varepsilon > 0~\exists \delta > 0~\text{ se } |x - x_0| < \delta \implies |f_{n_0}(x) - f_{n_0}(x_0)| < \varepsilon
  $$

  quindi, fissato $\varepsilon > 0$, se $|x - x_0| < \delta$:

  \begin{align*}
    |f(x) - f(x_0)| &= |f(x) - f_{n_0}(x) + f_{n_0}(x) - f_{n_0}(x_0) + f_{n_0}(x_0) - f(x_0)| \leqslant \\
                    &\leqslant |f(x) - f_{n_0}(x)| + |f_{n_0}(x) - f_{n_0}(x_0)| + |f_{n_0}(x_0) - f(x_0)| < 3 \varepsilon
  \end{align*}
\end{proof}


\begin{thm}[Convergenza uniforme e continuità]
  Sia $f_n: [a, b] \to \mathbb{R}$ una successioni di funzioni, se $f_n \in R(a, b)~\forall n$ e 
  $f_n \uniformto f$ $\implies$ $f \in R(a, b)$ e:

  $$
    \lim_{n \to \infty} \int_{a}^{b}|f_n(x) - f(x)|dx = 0
  $$
\end{thm}

\begin{proof}
  Si dimostrerà solo il secondo punto. Si ha che:

  $$
    \int_{a}^{b}|f_n(x) - f(x)|dx \leqslant \int_{a}^{b}\sup_{x \in [a, b]}|f_n(x) - f(x)|dx = (b - a)\sup_{x \in [a, b]}|f_n(x) - f(x)| \to 0
  $$

  dove nell'ultimo passaggio si è usata la convergenza uniforme di $f_n$.
\end{proof}


\begin{thm}[Convergenza uniforme e derivabilità]
  Sia $f_n: [a, b] \to \mathbb{R}$ una successione di funzioni derivabili. Si supponga che esista $g: [a, b] \to \mathbb{R}$ t.che 
  $f'_n \uniformto g$ e che esista un $x_0 \in [a, b]$ t.che $f_n(x_0)$ converge. Allora $f_n \uniformto f$, dove $f$ è 
  derivabile e:

  $$
    f' = g
  $$
  $$
    \lim f'_n = \left(\lim f_n\right)'
  $$
\end{thm}

\begin{proof}
  Si dimostrerà il caso in cui $f_n \in \mathcal{C}'([a, b])$. Si supponga inoltre che 
  $x_0 = a$. $f_n$ si può scrivere come (dato che $f'_n$ è continua e quindi integrabile):

  $$
    f_n(x) = f_n(a) + \int_{a}^{x}f_n'(t)dt
  $$

  si definisca allora:

  $$
    l := \lim f_n(a)
  $$

  e:

  $$
    f(x) := l + \int_{a}^{x}g(t)dt
  $$

  si ha che $f$ è continua e derivabile. Inoltre:

  $$
    f'(x) = g'(x)
  $$

  inoltre:

  \begin{align*}
    |f_n(x) - f(x)| &= |f_n(a) + \int_{a}^{x}f_n'(t)dt - l - \int_{a}^{x}g(t)dt| \leqslant \\
                    &\leqslant |f_n(a) - l| + |\int_{a}^{x}f_n'(t) - g(t)dt| \leqslant \\
                    &\leqslant |f_n(a) - l| + \int_{a}^{x}|f_n'(t) - g(t)|dt \leqslant \\
                    &\leqslant |f_n(a) - l| + \int_{a}^{b}|f_n'(t) - g(t)|dt
  \end{align*}

  di conseguenza:

  $$
    \sup_{x \in [a, b]}|f_n(x) - f(x)| \leqslant |f_n(a) - l| + \int_{a}^{b}|f_n'(t) - g(t)|dt \to 0
  $$

  la tesi segue per confronto.
\end{proof}


\section{Serie di funzioni}

\begin{define}[Serie di funzioni]
  Sia $\{f_n\}$ una successione di funzioni. Si definisca la successione di funzioni $s_n$ (detta successione delle somme parziali) come:

  $$
    s_n := \sum_{k=1}^{n}f_n(x)
  $$

  si definisce \textbf{serie di funzioni} di termine generale $f_n$, se esiste, la funzione $s: \mathfrak{I} \to \mathbb{R}$:

  $$
    s := \lim s_n
  $$
\end{define}

\begin{notation}
  $\sum f_n$, $\displaystyle \sum_{n=1}^{\infty}f_n$, $\displaystyle \sum_{n=1}^{\infty}f_n(x)$
\end{notation}


\subsection{Convergenza totale}

\begin{define}[Convergenza totale]
  Sia $\sum f_n$ una serie di funzioni, con $f_n: \mathfrak{I} \to \mathbb{R}$, si dice che $\sum f_n$ \textbf{converge totalmente} se esiste una successione numerica 
  $M_n \geqslant 0$ t.che:

  \begin{itemize}
    \item $\displaystyle \sup_{x \in \mathfrak{I}}|f_n(x)| \leqslant M_n$
    \item $\sum M_n$ converge
  \end{itemize}
\end{define}