\chapter{Successioni e serie di funzioni}


\section{Successioni di funzioni}


\begin{defineimp}[Successione di funzioni]
  Sia $f_n: \mathfrak{I} \to \mathbb{R}$ una funzione reale, con $n \in \mathbb{N}$, la successione 
  $\{f_n\}$ si dice \textbf{successione di funzioni}.
\end{defineimp}

\begin{notation}
  $\{f_n\}$, $f_n$, $\{f_n\}_{n \in \mathbb{N}}$
\end{notation}

\begin{obs}
  Fissato $x_0 \in \mathfrak{I}$, $f_n(x_0)$ è una successione numerica.
\end{obs}

\begin{defineimp}[Convergenza puntuale]
  Sia $f_n$ una successione di funzioni e sia $f: \mathfrak{I} \to \mathbb{R}$, si dice che 
  $f_n$ \textbf{converge puntualmente} a $f$ se:

  $$
    \forall x_0 \in \mathfrak{I}~f_n(x_0) \to f(x_0)
  $$
\end{defineimp}


\begin{es}[I]
  Sia $\mathfrak{I} = [0, 1]$ e sia $f_n$:

  $$
    f_n(x) = x^n
  $$

  $f_n$ converge puntualmente a:

  $$
    f(x) =  \begin{cases*}
              0~\text{ se }~x \in [0, 1) \\
              1~\text{ se }~x = 1
            \end{cases*}
  $$
\end{es}

\begin{center}
  \begin{tikzpicture}
  \begin{axis}[
    width=10cm,
    height=10cm,
    xmin=-0.1, xmax=1.1,
    ymin=-0.1, ymax=1.1,
    axis lines=middle,
    axis line style={->},
    xtick=\empty,
    ytick=\empty,
  ]

  \addplot[domain=0:1, smooth, very thick, cmapteal-100]
      {x};

  \addplot[domain=0:1, smooth, very thick, cmapteal-200]
      {x^2};

  \addplot[domain=0:1, smooth, very thick, cmapteal-300]
      {x^3};

  \addplot[domain=0:1, smooth, very thick, cmapteal-400]
      {x^4};

  \addplot[domain=0:1, smooth, very thick, cmapteal-500]
      {x^8};

  \addplot[domain=0:1, smooth, very thick, cmapteal-600]
      {x^32};

  % Labels
  \node[below] at (axis cs:1,0) {$1$};
  \node[left] at (axis cs:0,1) {$1$};
  % \node[below] at (axis cs:3.44,0) {$c$};


  \end{axis}
  \end{tikzpicture}
\end{center}

\begin{es}[II]
  Sia $\mathfrak{I} = [0, 1]$ e sia $f_n$:

  $$
    f_n(x) = \dfrac{\sin{(nx)}}{\sqrt{n}}
  $$

  $$
    f_n(x) \to f(x) = 0
  $$

  $$
    f'_n(x) = \sqrt{n}\cos{(nx)}
  $$
  $$
    f'_n(x) \to +\infty
  $$
\end{es}

\begin{es}[III]
  Sia $\mathfrak{I} = [0, 1]$ e sia $f_n$ così definita:

  \begin{center}
    \begin{tikzpicture}
    \begin{axis}[
      width=10cm,
      height=10cm,
      xmin=-0.1, xmax=1.1,
      ymin=-0.7, ymax=6.5,
      axis lines=middle,
      axis line style={->},
      xtick=\empty,
      ytick=\empty,
    ]

    \addplot[domain={1/3}:1, smooth, ultra thick, tyrian]
        {0};

    \draw[smooth, ultra thick, tyrian] (0, 0) -- ({1/(2*3)}, {2*3}) -- ({1/3}, 0);



    % \addplot[domain=0:5, smooth, thick, teal]
    %     {((e^(cos(deg(5)))*sqrt(5) - e^(cos(deg(1))))/(5-1))*(x - 1) + e^(cos(deg(1)))};

    % \addplot[domain=0.5:5.5, smooth, thick, amber]
    %     {((e^(cos(deg(5)))*sqrt(5) - e^(cos(deg(1))))/(5-1))*(x - 3.44) + e^(cos(deg(3.44)))*sqrt(3.44)};

    % Labels
    \node[below] at (axis cs:1,0) {$1$};
    \node[left] at (axis cs:0,1) {$1$};
    \node[below] at (axis cs:{1/3},0) {$\dfrac{1}{n}$};
    \node[left] at (axis cs:0,{2*3}) {$2n$};

    % \draw[dashed, gray] (1, 0) -- (X);
    % \draw[dashed, gray] (5, 0) -- (Y);
    % \draw[dashed, gray] (3.44, 0) -- (C);

    \end{axis}
    \end{tikzpicture}
  \end{center}

  $$
    f_n(x) \to f(x) = 0
  $$

  $$
    \int_{0}^{1}f_n(x)dx = 1~\forall n \in \mathbb{N}_0
  $$
  $$
    \lim_{n \to \infty} \int_{0}^{1}f_n(x)dx = 1 \neq \int_{0}^{1}\lim_{n \to \infty} f(x)dx = 0
  $$
\end{es}


\begin{prop}
  Se $f_n$ è continua $\forall n$ $\centernot \implies$ $f$ è continua.
\end{prop}

\begin{prop}
  Se $f_n$ è derivabile $\forall n$ $\centernot \implies$ $f$ è derivabile.
\end{prop}

\begin{prop}
  Se $f_n \in R(\mathfrak{I})$ $\forall n$ $\centernot \implies$ $f \in R(\mathfrak{I})$.
\end{prop}

\begin{es}
  Sia $f_n$ così definita nell'intervallo $[0, 1]$:

  $$
    f_n(x) =  \begin{cases*}
                \dfrac{1}{x}~\text{ se }~x \geqslant \dfrac{1}{n} \\
                0~\text{ se }~x < \dfrac{1}{n} \\
              \end{cases*}
  $$

  $$
    f_n(x) \to f(x) = \dfrac{1}{x}
  $$

  ma $f$ non è limitata in $[0, 1]$, quindi non è integrabile.
\end{es}

\begin{prop}
  Se $f_n \in R(\mathfrak{I})$ $\forall n$ e $f \in R(\mathfrak{I})$ $\centernot \implies$ $\lim_{n \to \infty}\int_{\mathfrak{I}}f_n(x)dx = \int_{\mathfrak{I}}f(x)dx = \int_{\mathfrak{I}}\lim_{n \to \infty}f_n(x)dx$.
\end{prop}


\subsection{Convergenza uniforme}

\begin{defineimp}[Convergenza uniforme]
  Sia $\{f_n\}$ una successione di funzioni e sia $f: \mathfrak{I} \to \mathbb{R}$, $f_n$ 
  \textbf{converge uniformemente} a $f$ se:

  $$
    \forall \varepsilon > 0~\exists n_0 \in \mathbb{N}~\colon~|f_n(x) - f(x)| < \varepsilon~\forall n \geqslant n_0~\forall x \in \mathfrak{I}
  $$
\end{defineimp}

\begin{center}
  \begin{tikzpicture}
  \begin{axis}[
    width=10cm,
    height=8cm,
    xmin=-1, xmax=7,
    ymin=-5, ymax=5,
    axis lines=middle,
    axis line style={->},
    xtick=\empty,
    ytick=\empty,
  ]

  % Convex function (example)
  \addplot[domain=0:6.28, smooth, ultra thick, dashed, cmapteal-600]
      {sin(deg(x))};

  \addplot[domain=0:6.28, smooth, very thick, cmapteal-300]
      {sin(deg(x))-3*sin(deg(x))*(-1)^1/(1)^2};

  \addplot[domain=0:6.28, smooth, very thick, cmapteal-500]
      {sin(deg(x))-3*sin(deg(x))*(-1)^2/(2)^2};

  \addplot[domain=0:6.28, smooth, very thick, cmapteal-400]
      {sin(deg(x))-3*sin(deg(x))*(-1)^3/(3)^2};

  % \addplot[domain=0:5, smooth, thick, teal]
  %     {((e^(cos(deg(5)))*sqrt(5) - e^(cos(deg(1))))/(5-1))*(x - 1) + e^(cos(deg(1)))};

  % \addplot[domain=0.5:5.5, smooth, thick, amber]
  %     {((e^(cos(deg(5)))*sqrt(5) - e^(cos(deg(1))))/(5-1))*(x - 3.44) + e^(cos(deg(3.44)))*sqrt(3.44)};

  % Labels
  % \node[below] at (axis cs:1,0) {$a$};
  % \node[below] at (axis cs:5,0) {$b$};
  % \node[below] at (axis cs:3.44,0) {$c$};

  % \draw[dashed, gray] (1, 0) -- (X);
  % \draw[dashed, gray] (5, 0) -- (Y);
  % \draw[dashed, gray] (3.44, 0) -- (C);

  \end{axis}
  \end{tikzpicture}
\end{center}

\begin{notation}
  $f_n \uniformto f$
\end{notation}

\begin{obs}
  La scelta di $n_0$ non dipende da $x$.
\end{obs}

\begin{obs}
  $$
    f_n \uniformto f~\iff~\lim_{n \to \infty}\sup_{x \in \mathfrak{I}}|f_n(x) - f(x)| = 0
  $$
\end{obs}


\begin{thmimp}[Convergenza uniforme e continuità]
  Sia $f_n: \mathfrak{I} \to \mathbb{R}$ una successioni di funzioni, se $f_n$ è continua $\forall n$ e 
  $f_n \uniformto f$ $\implies$ $f$ è continua.
\end{thmimp}

\begin{proof}
  Dato che $f_n$ converge uniformemente a $f$, si ha che, fissato $x_0 \in \mathfrak{I}$:

  $$
    \forall \varepsilon > 0~\exists n_0 \in \mathbb{N}~\colon~|f_n(x) - f(x)| < \varepsilon~\forall n \geqslant n_0~\forall x \in \mathfrak{I}
  $$

  dato che $f_n$ è continua:

  $$
    \forall \varepsilon > 0~\exists \delta > 0~\text{ se } |x - x_0| < \delta \implies |f_{n_0}(x) - f_{n_0}(x_0)| < \varepsilon
  $$

  quindi, fissato $\varepsilon > 0$, se $|x - x_0| < \delta$:

  \begin{align*}
    |f(x) - f(x_0)| &= |f(x) - f_{n_0}(x) + f_{n_0}(x) - f_{n_0}(x_0) + f_{n_0}(x_0) - f(x_0)| \leqslant \\
                    &\leqslant |f(x) - f_{n_0}(x)| + |f_{n_0}(x) - f_{n_0}(x_0)| + |f_{n_0}(x_0) - f(x_0)| < 3 \varepsilon
  \end{align*}
\end{proof}


\begin{thmimp}[Convergenza uniforme e integrabilità]
  Sia $f_n: [a, b] \to \mathbb{R}$ una successioni di funzioni, se $f_n \in R(a, b)~\forall n$ e 
  $f_n \uniformto f$ $\implies$ $f \in R(a, b)$ e:

  $$
    \lim_{n \to \infty} \int_{a}^{b}|f_n(x) - f(x)|dx = 0
  $$
\end{thmimp}

\begin{proof}
  Si dimostrerà solo il secondo punto. Si ha che:

  $$
    \int_{a}^{b}|f_n(x) - f(x)|dx \leqslant \int_{a}^{b}\sup_{x \in [a, b]}|f_n(x) - f(x)|dx = (b - a)\sup_{x \in [a, b]}|f_n(x) - f(x)| \to 0
  $$

  dove nell'ultimo passaggio si è usata la convergenza uniforme di $f_n$.
\end{proof}


\begin{thmimp}[Convergenza uniforme e derivabilità]
  Sia $f_n: [a, b] \to \mathbb{R}$ una successione di funzioni derivabili. Si supponga che esista $g: [a, b] \to \mathbb{R}$ t.che 
  $f'_n \uniformto g$ e che esista un $x_0 \in [a, b]$ t.che $f_n(x_0)$ converge. Allora $f_n \uniformto f$, dove $f$ è 
  derivabile e:

  $$
    f' = g
  $$
  $$
    \lim f'_n = \left(\lim f_n\right)'
  $$
\end{thmimp}

\begin{proof}
  Si dimostrerà il caso in cui $f_n \in \mathcal{C}'([a, b])$. Si supponga inoltre che 
  $x_0 = a$. $f_n$ si può scrivere come (dato che $f'_n$ è continua e quindi integrabile):

  $$
    f_n(x) = f_n(a) + \int_{a}^{x}f_n'(t)dt
  $$

  si definisca allora:

  $$
    l := \lim f_n(a)
  $$

  e:

  $$
    f(x) := l + \int_{a}^{x}g(t)dt
  $$

  si ha che $f$ è continua e derivabile. Inoltre:

  $$
    f'(x) = g'(x)
  $$

  inoltre:

  \begin{align*}
    |f_n(x) - f(x)| &= |f_n(a) + \int_{a}^{x}f_n'(t)dt - l - \int_{a}^{x}g(t)dt| \leqslant \\
                    &\leqslant |f_n(a) - l| + |\int_{a}^{x}f_n'(t) - g(t)dt| \leqslant \\
                    &\leqslant |f_n(a) - l| + \int_{a}^{x}|f_n'(t) - g(t)|dt \leqslant \\
                    &\leqslant |f_n(a) - l| + \int_{a}^{b}|f_n'(t) - g(t)|dt
  \end{align*}

  di conseguenza:

  $$
    \sup_{x \in [a, b]}|f_n(x) - f(x)| \leqslant |f_n(a) - l| + \int_{a}^{b}|f_n'(t) - g(t)|dt \to 0
  $$

  la tesi segue per confronto.
\end{proof}


\section{Serie di funzioni}

\begin{defineimp}[Serie di funzioni]
  Sia $\{f_n\}$ una successione di funzioni. Si definisca la successione di funzioni $s_n$ (detta successione delle somme parziali) come:

  $$
    s_n := \sum_{k=1}^{n}f_n(x)
  $$

  si definisce \textbf{serie di funzioni} di termine generale $f_n$, se esiste, la funzione $s: \mathfrak{I} \to \mathbb{R}$:

  $$
    s := \lim s_n
  $$
\end{defineimp}

\begin{notation}
  $\sum f_n$, $\displaystyle \sum_{n=1}^{\infty}f_n$, $\displaystyle \sum_{n=1}^{\infty}f_n(x)$
\end{notation}


\begin{thm}[Convergenza uniforme e continuità per le serie di funzioni]
  Se $f_n$ è continua $\forall n$ e $\sum f_n \uniformto s$ $\implies$ $s$ è continua
\end{thm}

\begin{thm}[Convergenza uniforme e integrabilità per le serie di funzioni]
  Sia $f_n: [a, b] \to \mathbb{R}$, se $f_n \in R(a, b)$ $\forall n$ e $\sum f_n \uniformto s$ $\implies$ $s \in R(a, b)$. Inoltre:

  $$
    \sum_{n=1}^{\infty}\int_{a}^{b}f_n(x)dx = \int_{a}^{b}\sum_{n=1}^{\infty}f_n(x)dx
  $$
\end{thm}

\begin{thm}[Convergenza uniforme e derivabilità per le serie di funzioni]
  Sia $f_n: [a, b] \to \mathbb{R}$ t.che $f_n$ è derivabile $\forall n$, se $\exists x_0 \in [a, b]$ t.che $\sum_{n=1}^{\infty}f_n(x_0)$ converge e 
  $\sum f'_n \uniformto s': [a, b] \to \mathbb{R}$ $\implies$ $\sum f_n \uniformto s: [a, b] \to \mathbb{R}$ e $\sum f_n$ è derivabile. Inoltre: 

  $$
    \sum_{n=1}^{\infty}f_n'(x) = \left(\sum_{n=1}^{\infty}f_n(x)\right)'
  $$
\end{thm}


\subsection{Convergenza totale}

\begin{defineimp}[Convergenza totale]
  Sia $\sum f_n$ una serie di funzioni, con $f_n: \mathfrak{I} \to \mathbb{R}$, si dice che $\sum f_n$ \textbf{converge totalmente} se esiste una successione numerica 
  $M_n \geqslant 0$ t.che:

  \begin{itemize}
    \item $\displaystyle \sup_{x \in \mathfrak{I}}|f_n(x)| \leqslant M_n$
    \item $\sum M_n$ converge
  \end{itemize}
\end{defineimp}


\begin{thm}
  Se $\sum f_n$ converge totalmente $\implies$ $\sum f_n$ converge uniformemente
\end{thm}

\begin{proof}
  Sia $s_n$ la successione delle somme parziali. Si noti inoltre che, se $\sum_{k=1}^{\infty} M_k$ converge, allora:

  $$
    \sum_{k=n+1}^{\infty}M_k \underset{n \to +\infty}{\to} 0
  $$

  si ha quindi che:

  \begin{align*}
    |s_n(x) - s(x)| &= \left|\sum_{k=1}^{n}f_n(x) - \sum_{k=1}^{\infty}f_n(x)\right| = \left|\sum_{k=n+1}^{\infty}f_n(x)\right| \leqslant \\
                    &\leqslant \sum_{k=n+1}^{\infty}|f_n(x)| \leqslant \sum_{k=n+1}^{\infty}M_k~~~\forall x \in \mathfrak{I}
  \end{align*}

  dunque:

  $$
    \sup_{x \in \mathfrak{I}}|s_n(x) - s(x)| \leqslant \sum_{k=n+1}^{\infty}M_k \underset{n \to +\infty}{\to} 0
  $$
\end{proof}


\subsection{Serie di potenze}

\begin{defineimp}[Serie di potenze]
  Sia $\{a_n\}$ una successione numerica, si definisce \textbf{serie di potenze} la serie:

  $$
    \sum_{n=0}^{\infty}a_n x^n
  $$
\end{defineimp}

\begin{obs}
  Se $x = 0$, la serie converge a $a_0$.
\end{obs}

\begin{defineimp}[Raggio di convergenza]
  Sia $\sum a_n x^n$ una serie di potenze, si dice \textbf{raggio di convergenza} della serie:

  $$
    R := \dfrac{1}{L} \in [0, +\infty]
  $$

  dove:

  $$
    L := \limsup \sqrt[n]{|a_n|} \in [0, +\infty]
  $$
\end{defineimp}

\begin{obs}
  Se $a_n$ è polinomiale, $R = 1$.
\end{obs}


\begin{thmimp}
  Sia $\sum a_n x^n$ una serie di potenze e sia $R$ il suo raggio di convergenza, allora se $R > 0$:

  \begin{itemize}
    \item la serie converge puntualmente su $(-R, R)$
    \item la serie converge totalmente su ogni $[-r, r]$ con $0 < r < R$
    \item la serie non converge in alcun $x \in [-R, R]^C$
  \end{itemize}
\end{thmimp}

\begin{obs}
  Se $R = +\infty$ si ha convergenza totale su ogni intervallo chiuso.
\end{obs}

\begin{obs}
  Se $R = 0$, la serie converge solo in $x = 0$.
\end{obs}

\begin{proof}\leavevmode
  \begin{itemize}
    \item Il primo punto segue dal secondo
    \item Sia $0 < r < R$ e sia $x \in [-r, r]$, allora:
          $$
            |a_n x^n| \leqslant |a_n| r^n
          $$

          inoltre, per il criterio della radice la serie $\sum |a_n|r^n$ converge:

          $$
            \limsup \sqrt[n]{|a_n|r^n} = \limsup r \sqrt[n]{|a_n|} = \dfrac{r}{R} < 1
          $$

    \item Sia $|x| > R$:

          $$
            \limsup \sqrt[n]{|a_n x^n|} = \dfrac{|x|}{R} > 1
          $$

          quindi esiste una sottosuccessione t.che $|a_{n_k} x^{n_k}| > 1$, quindi $a_n x^n \not \to 0$.
  \end{itemize}
\end{proof}

\begin{obs}
  Si può sostituire $x$ con $f(x)$:

  $$
    \sum a_n [f(x)]^n
  $$

  ed ottenere il raggio di convergenza ponendo $y := f(x)$ e imponendo $y \in (-R, R)$.
\end{obs}


\begin{lemma}[Lemma di Abel]
  Sia $\sum a_n x^n$ una serie di potenze di raggio di convergenza $0 < R < +\infty$, se $\sum a_n R^n$ converge allora 
  la serie converge uniformemente su $[-r, R]~\forall 0 < r < R$. Lo stesso vale per l'altro estremo.
  Di conseguenza, se $\sum a_n x^n$ converge su $x = -R$ e $x = R$ si ha convergenza uniforme su $[-R, R]$.
\end{lemma}



\begin{thm}[Continuità per le serie di potenze]
  Sia $\sum a_n x^n$ una serie di potenze di raggio di convergenza $R > 0$ e sia:

  $$
    s(x) := \sum_{n=0}^{\infty}a_n x^n
  $$

  allora $s$ è continua su $(-R, R)$. Inoltre, se la serie converge su $x = R$, 
  $s$ è continua su $(-R, R]$. Lo stesso vale per l'altro estremo.
\end{thm}

\begin{thm}[Integrabilità per le serie di potenze]
  Sia $\sum a_n x^n$ una serie di potenze di raggio di convergenza $R > 0$ e sia:

  $$
    s(x) := \sum_{n=0}^{\infty}a_n x^n
  $$

  allora $s$ è integrabile su ogni $[a, b]$, con $-R < a < b < R$. Inoltre:

  $$
    \int_{a}^{b}s(x) dx = \sum_{n=0}^{\infty} \int_{a}^{b}a_n x^n dx = \sum_{n=0}^{\infty}\dfrac{a_n}{n+1}(b^{n+1}-a^{n+1})
  $$
  
  Inoltre, se la serie converge su $x = R$, 
  $s$ è integrabile su $-R < a < b \leqslant R$. Lo stesso vale per l'altro estremo.
\end{thm}

\begin{lemma}
  Se $\sum_{n=0}^{\infty} a_n x^n$ ha raggio di convergenza $R > 0$ $\implies$ $\sum_{n=1}^{\infty} n a_n x^{n-1}$ ha raggio di convergenza $R$.
\end{lemma}

\begin{proof}
  Si ha che:

  $$
    \sum_{n=1}^{\infty} n a_n x^{n-1} = \sum_{n=0}^{\infty}(n+1)a_{n+1}x^{n}
  $$

  inoltre:

  $$
    \limsup \sqrt[n]{(n + 1)|a_{n+1}|} = \limsup \sqrt[n]{n+1}\left(|a_{n+1}|^{\frac{1}{n+1}}\right)^{\frac{n+1}{n}} = L
  $$
\end{proof}

\begin{thm}[Derivabilità per le serie di potenze]
  Sia $\sum a_n x^n$ una serie di potenze di raggio di convergenza $R > 0$ e sia:

  $$
    s(x) := \sum_{n=0}^{\infty}a_n x^n
  $$

  allora $s$ è derivabile su $(-R, R)$. Inoltre:

  $$
    s'(x) = \sum_{n=1}^{\infty}n a_n x^{n-1}
  $$
\end{thm}


\begin{obs}
  $s \in \mathcal{C}^{\infty}(-R, R)$. Inoltre:

  $$
    s^{(k)}(x) = \sum_{n=k}^{\infty}n(n-1)(n-2)\dots(n-k+1)a_n x^{n-k}~~~\forall k=0,1,\dots
  $$

  $$
    s^{(k)}(0) = k!\cdot a_k
  $$

  $$
    s(x) = \sum_{n=0}^{\infty}\dfrac{s^{(n)}(0)}{n!}x^n
  $$
\end{obs}


\begin{es}
  $$
    s(x) := \sum_{n=0}^{\infty}\dfrac{(-1)^n}{2n + 1}x^{2n + 1}
  $$

  si ha che $R = 1$. Sia $x \in (-1, 1)$, si ha che:

  $$
    s'(x) = \sum_{n=0}^{\infty}(-1)^n x^{2n} = \sum_{n=0}^{\infty}(-x^2)^n = \dfrac{1}{1 + x^2}
  $$

  quindi, per il teorema fondamentale del calcolo integrale:

  $$
    s(x) - \equalto{s(0)}{0} = \arctan{x} \implies s(x) = \arctan{x}
  $$

  si ha inoltre che $s$ converge puntualmente per $x = 1$. Quindi, per il lemma di Abel:

  $$
    s(1) = \arctan{1} = \dfrac{\pi}{4}
  $$
\end{es}


\begin{es}
  Si vuole calcolare:

  $$
    \sum_{n=0}^{\infty}\dfrac{1}{2^n (n + 2)}
  $$

  sia allora:

  $$
    s(x) := \sum_{n=0}^{\infty}\dfrac{x^n}{n+2}
  $$

  si ha che $s$ ha raggio di convergenza $R = 1$. Sia quindi $x \in (-1, 1)$, con $x \neq 0$:

  $$
    s(x)x^2 = \sum_{n=0}^{\infty}\dfrac{x^{n+2}}{n+2}
  $$

  $$
    \left(s(x)x^2\right)' = \sum_{n=0}^{\infty}x^{n+1} = x\sum_{n=0}^{\infty}x^{n} = \dfrac{x}{1 - x}
  $$

  inoltre:

  $$
    \int_{0}^{x}\dfrac{t}{1 - t}dt = -\int_{0}^{x}dt + \int_{0}^{1}\dfrac{1}{1 - t}dt = -x - \log{(1 - x)}
  $$

  quindi:

  $$
    s(x)x^2 = -x - \log{(1 - x)} \implies s(x) = -\dfrac{1}{x} - \dfrac{\log{(1 - x)}}{x^2}
  $$

  si noti che, pur non essendo definita in $x=0$, la funzione trovata può essere estesa per continuità in tale punto (dove infatti la serie converge a $\dfrac{1}{2}$). Si ha quindi che:

  $$
    \sum_{n=0}^{\infty}\dfrac{1}{2^n (n + 2)} = s\left(\dfrac{1}{2}\right) = -2 + 4\log{2}
  $$
\end{es}


\begin{es}
  Si vuole calcolare:

  $$
    s(x) = \sum_{n=1}^{\infty}nx^n
  $$

  il raggio di convergenza di tale serie di funzioni è $R = 1$. Sia $x \in (-1, 1)$ e sia $F$ così definita:

  $$
    F(x) := \sum_{n=1}^{\infty}x^n = \dfrac{1}{1 - x}
  $$

  si ha che:

  $$
    F'(x) = \sum_{n=1}^{\infty}nx^{n-1} \implies xF'(x) = \sum_{n=1}^{\infty}nx^{n} = s(x)
  $$

  $$
    s(x) = x F'(x) = \dfrac{x}{(1 - x)^2}
  $$
\end{es}