\chapter{Integrali}

\section{L'integrale di Riemann}

\section{Costruzione dell'integrale di Riemann}

Sia $f: [a, b] \to \mathbb{R}$ una funzione limitata.

\begin{center}
  \begin{tikzpicture}[scale=1.1]

  % Main line
  \draw[thick] (0,0) -- (10,0);

  % Endpoints
  \node[below] at (0,0) {$a$};
  \node[below] at (10,0) {$b$};

  % Number of subdivisions
  \def\n{5} % change n here

  % Subdivision points
  \foreach \k in {1,...,\numexpr\n-2} {
    \draw[thick] ({10*\k/\n},0) -- ({10*\k/\n},0.15);
    \node[below, cmapteal-300] at ({10*\k/\n},0) {$x_{\k}$};
  }

  \draw[thick] ({10*(\numexpr\n-1)/\n},0) -- ({10*(\numexpr\n-1)/\n},0.15);
  \node[below, cmapteal-300] at ({10*(\numexpr\n-1)/\n},0) {$x_{n-1}$};

  % Interval braces and labels
  \foreach \k in {1,...,\numexpr\n-1} {
    \draw[decorate,decoration={brace,mirror,amplitude=4pt}]
      ({10*(\k-1)/\n},-0.5) -- ({10*\k/\n},-0.5)
      node[midway,below=6pt] {$\mathfrak{I}_{\k}$};
  }

  \draw[decorate,decoration={brace,mirror,amplitude=4pt}]
      ({10*(\numexpr\n-1)/\n},-0.5) -- ({10*((\numexpr\n-1)+1)/\n},-0.5)
      node[midway,below=6pt] {$\mathfrak{I}_{n}$};

  % Length annotation
  \draw[<->] (0,1) -- ({10/\n},1);
  \node[above] at ({5/\n},1) {$\frac{b-a}{n}$};

  \end{tikzpicture}
\end{center}

Si fissi $n \in \mathbb{N}$, ogni intervallo ha lunghezza $\dfrac{b-a}{n}$.

$$
  x_k = a + \dfrac{b-a}{n}k
$$

$$
  \mathfrak{I}_k = [x_{k-1}, x_k]~~~\forall k=1,2\dots,n
$$

Dato che $f$ è limitata, $\forall k=1,2\dots,n$ si ha che $\displaystyle \inf_{\mathfrak{I}_k} f$ e $\sup_{\mathfrak{I}_k} f$ sono finiti.

Si defiscano allora:

\begin{define}[Somme inferiori]
  Si definisce \textbf{somma inferiore} di ordine $n$ la successione:

  $$
    \sigma_n = \dfrac{b-a}{n}\sum_{k=1}^n \inf_{\mathfrak{I}_k} f
  $$
\end{define}

\begin{define}[Somme superiori]
  Si definisce \textbf{somma superiore} di ordine $n$ la successione:

  $$
    S_n = \dfrac{b-a}{n}\sum_{k=1}^n \sup_{\mathfrak{I}_k} f
  $$
\end{define}

\begin{obs}
  Dato che $\sigma_n$ è l'area di un plurirettangolo sotto il grafico della funzione, 
  mentre $S_n$ è l'area di un plurirettangolo sopra il grafico della funzione, si ha che:

  $$
    \sigma_n \leqslant S_n~\forall n~\forall m
  $$
\end{obs}

\begin{obs}[$\diamondsuit$]
  Non è detto che $S_n$ sia monotona decrescente.
\end{obs}

\begin{es}
  $$
    f(x) =  \begin{cases*}
              1~\text{ se } x \in \left[0, \dfrac{1}{2}\right) \\
              0~\text{ se } x \in \left[\dfrac{1}{2}, 1\right] \\
            \end{cases*}
  $$

  $$
    S_2 = \dfrac{1}{2}
  $$
  $$
    S_3 = \dfrac{2}{3}
  $$
\end{es}


\begin{define}
  $S := \lim S_n$, $\sigma := \lim \sigma_n$
\end{define}

\begin{obs}
  Per il teorema della permanenza del segno, si ha che:

  $$
    \sigma_n \leqslant S
  $$
  $$
    \sigma \leqslant S
  $$
  $$
    \sigma_n \leqslant \sigma \leqslant S \leqslant S_n~\forall n
  $$
\end{obs}


\begin{defineimp}[Integrabilità secondo Riemann]
  $f$ si dice \textbf{integrabile secondo Riemann} su $[a, b]$ se 
  
  $$
    \lim \sigma_n = \lim S_n = S = \sigma
  $$
\end{defineimp}

\begin{notation}
  $R(a, b)$, insieme delle funzioni integrabili secondo Riemann su $[a, b]$
\end{notation}

\begin{obs}
  Se una funzione $f \in R(a, b)$, $f$ è limitata.
\end{obs}


\begin{defineimp}[Integrale di Riemann]
  Se $f \in R(a, b)$, $S = \sigma$ è detto \textbf{integrale di Riemann} di $f$ su $[a, b]$.
\end{defineimp}

\begin{notation}
  $\displaystyle \int_{a}^{b} f(x)dx$, $\displaystyle \int_{[a, b]} f(x)dx$, $\displaystyle \int_{a}^{b}f$
\end{notation}

\begin{notation}
  $\Delta_{\mathfrak{I}_k}f := \sup_{\mathfrak{I}_k}f - \inf_{\mathfrak{I}_k}f$
\end{notation}

\begin{define}
  $$
    \omega_n(f) := S_n - \sigma_n = \dfrac{b-a}{n}\sum_{k=1}^{n} \sup_{\mathfrak{I}_k}f - \inf_{\mathfrak{I}_k}f = \dfrac{b-a}{n}\sum_{k=1}^{n} \Delta_{\mathfrak{I}_k}f
  $$
\end{define}

\begin{thm}
  $f \in R(a, b) \iff \omega_n(f) \to 0$
\end{thm}

\begin{proof}\leavevmode
  \begin{itemize}
    \item ($\Rightarrow$): Dato che $f \in R(a, b)$, $S_n \to S$, $\sigma_n \to \sigma = S$. Quindi:
          $$
            \omega_n(f) = S_n - \sigma_n \to S - S = 0
          $$
    \item ($\Leftarrow$): Si ha che:
          $$
            \sigma_n \leqslant \sigma \leqslant S \leqslant S_n
          $$
          quindi:

          $$
            0 \leqslant \sigma - \sigma_n \leqslant S - S_n \leqslant S_n - \sigma_n = \omega_n(f) \to 0
          $$

          dunque per confronto, $\sigma_n \to \sigma$ e $S_n \to S$. Inoltre:

          $$
            0 \leqslant S - S_n \leqslant S - \sigma_n \leqslant S_n - \sigma_n
          $$

          quindi $\sigma_n \to S = \sigma$.
  \end{itemize}
\end{proof}


\begin{define}[Area]

\end{define}


\begin{define}[Funzione caratteristica]
  Dato un insieme $A \subset \mathbb{R}$, si definisce \textbf{funzione caratteristica} di $A$:

  $$
    \chi_A(x) := \begin{cases*}
                1~\text{ se } x \in A \\
                0~\text{ se } x \not \in A
              \end{cases*}  
  $$
\end{define}


\begin{thm}
  Siano $f, g: [a, b] \to \mathbb{R}$ t.che $f$ e $g$ differiscono in un numero finito 
  di punti. Allora, se $f \in R(a, b)$, $g \in R(a, b)$ e:

  $$
    \int_{a}^{b}f(x)dx = \int_{a}^{b}g(x)dx
  $$
\end{thm}

\begin{proof}

\end{proof}


\begin{thm}
  Esistono funzioni reali $f$ limitate t.che $f \not \in R(a, b)$.
\end{thm}

\begin{es}[Funzione di Dirichlet]
  Sia $D:[0, 1] \to \mathbb{R}$:

  $$
    D(x) = \chi_{[0, 1] \cap \mathbb{Q}}(x)
  $$

  si ha che $D \not \in R(a, b)$. Infatti, fissato $n \in \mathbb{N}$:

  $$
    S_n = \dfrac{1}{n}\left[\sup_{\mathfrak{I}_1}D + \sup_{\mathfrak{I}_2}D + \dots + \sup_{\mathfrak{I}_n}D\right]
  $$

  in ogni $\mathfrak{I}_k$, $\displaystyle \sup_{\mathfrak{I}_k}D = 1$ (per la densità di $\mathbb{Q}$ in $\mathbb{R}$). Quindi:

  $$
    S = 1
  $$

  ma 

  $$
    \sigma_n = \dfrac{1}{n}\left[\inf_{\mathfrak{I}_1}D + \inf_{\mathfrak{I}_2}D + \dots + \inf_{\mathfrak{I}_n}D\right] = 0 \to 0 = \sigma \neq S
  $$
\end{es}


\begin{define}
  Se $A \subset \mathbb{R}$ è limitato, preso $[-N, N] \supset A$ si può definire 
  una misura di $A$ come:

  $$
    m(A) = \int_{-N}^{N} \chi_{A}(x)dx
  $$
\end{define}

\begin{obs}
  Non tutti gli insiemi sono misurabili secondo questa misura.
\end{obs}


\begin{lemma}
  Siano $f, g: \mathfrak{I} \to \mathbb{R}$ limitate, si ha che:

  $$
    \sup_{\mathfrak{I}}(f + g) \leqslant \sup_{\mathfrak{I}}f + \sup_{\mathfrak{I}}g
  $$
  $$
    \inf_{\mathfrak{I}}(f + g) \geqslant \inf_{\mathfrak{I}}f + \inf_{\mathfrak{I}}g
  $$
\end{lemma}

\begin{proof}
  Fissato $x \in \mathfrak{I}$:

  $$
    f(x) \leqslant \sup_{\mathfrak{I}}f
  $$
  $$
    g(x) \leqslant \sup_{\mathfrak{I}}g
  $$
  $$
    f(x) + g(x) \leqslant \sup_{\mathfrak{I}}f + \sup_{\mathfrak{I}}g
  $$
  $$
    \sup_{\mathfrak{I}}(f + g) \leqslant \sup_{\mathfrak{I}}f + \sup_{\mathfrak{I}}g
  $$
  dove nell'ultimo passaggio si è utilizzato il teorema della permanenza del segno. In modo analogo 
  si può procedere per gli estremi inferiori.
\end{proof}


\begin{thmimp}[Linearità dell'integrale]
  Se $f, g \in R(a, b)$ e $\lambda \in \mathbb{R}$, allora:

  $$
    f + \lambda g \in R(a, b)
  $$
  $$
    \int_{a}^{b}f(x) + \lambda g(x) dx = \int_{a}^{b}f(x)dx + \lambda\int_{a}^{b}g(x) dx
  $$
\end{thmimp}

\begin{proof}
  Consideriamo il caso in cui $f(x) = 0$. Si vuole mostrare che:

  $$
    \int_{a}^{b} \lambda g(x)dx = \lambda \int_{a}^{b}g(x)dx
  $$

  Si ha che:

  $$
    \sigma_n(\lambda g) = \lambda \sigma_n(g)~\text{ se }~\lambda > 0
  $$
  $$
    \sigma_n(\lambda g) = \lambda S_n(g)~\text{ se }~\lambda < 0
  $$

  da cui segue la tesi (passando ai limiti). \\

  Consideriamo ora il caso in cui $f(x) \neq 0$. Fissato $\mathfrak{I}_k$, si ha che:

  $$
    \inf_{\mathfrak{I}_k}f + \inf_{\mathfrak{I}_k}g \leqslant \inf_{\mathfrak{I}_k}(f + g) \leqslant \sup_{\mathfrak{I}_k}(f + g) \leqslant \sup_{\mathfrak{I}_k}f + \sup_{\mathfrak{I}_k}g
  $$

  $$
    \sigma_n(f) + \sigma_n(g) \leqslant \sigma_n(f + g) \leqslant S_n(f + g) \leqslant S_n(f) + S_n(g)
  $$

  da cui segue la tesi per confronto.
\end{proof}


\begin{define}[Funzione costante a tratti]
  Una funzione $h: [a, b] \to \mathbb{R}$ è \textbf{costante a tratti} se 
  $\exists \mathfrak{I}_1, \mathfrak{I}_2, \dots, \mathfrak{I}_n$ intervalli in $[a, b]$ e 
  $\exists \lambda_1, \lambda_2, \dots, \lambda_n$ t.che:
  
  $$
    h(x) = \sum_{k=1}^{n}\lambda_k\chi_{\mathfrak{I}_k}(x)
  $$
\end{define}

\begin{prop}
  Se $\displaystyle h(x) = \sum_{k=1}^{n}\lambda_k \chi_{\mathfrak{I}_k}(x)$, ($\mathfrak{I}_k \subset [a, b]$), 
  allora $h \in R(a, b)$ e:

  $$
    \int_{a}^{b}h(x)dx = \sum_{k=1}^{n}\lambda_k \int_{a}^{b}\chi_{\mathfrak{I}_k}(x) = \sum_{k=1}^{n}\lambda_k \ell(\mathfrak{I}_k)
  $$
\end{prop}

\begin{thmimp}[Teorema del confronto per l'integrale di Riemann]
  Siano $f, g \in R(a, b)$, se $f \geqslant g$ ($f(x) \geqslant g(x)~\forall x \in [a, b]$), allora:

  $$
    \int_{a}^{b}f(x)dx \geqslant \int_{a}^{b}g(x)dx
  $$
\end{thmimp}

\begin{proof}
  Dato che $f \geqslant g$, si ha che:

  $$
    \sigma_n(f) \geqslant \sigma_n(g)
  $$
  
  la tesi segue dal teorema della permanenza del segno per le successioni.
\end{proof}

\begin{cor}
  Sia $f \in R(a, b)$ t.che $f \geqslant 0$, allora:

  $$
    \int_{a}^{b}f(x)dx \geqslant 0
  $$
\end{cor}

\begin{obs}
  Sia $f \in R(a, b)$, $f \geqslant 0$, $\int_{a}^{b}f(x)dx = 0$ $\centernot \implies$ $f(x) = 0$
\end{obs}

\begin{es}
  Basti prendere una $f$ che differisca da $0$ in un numero finito di punti
\end{es}


\begin{prop}
  Sia $f \in R(a, b)$, $f \geqslant 0$, $\int_{a}^{b}f(x)dx = 0$, $f$ continua $\implies$ $f(x) = 0$
\end{prop}

\begin{proof}

\end{proof}

\begin{obs}
  Non è necessario specificare che $f \in R(a, b)$ (dato che si vedrà in seguito che la continuità implica l'integrabilità).
\end{obs}


\begin{thmimp}[Disuguaglianza del modulo]
  Sia $f \in R(a, b)$, allora $|f| \in R(a, b)$ e:

  $$
    \left|\int_{a}^{b}f(x)dx\right| \leqslant \int_{a}^{b}\left|f(x)\right|dx
  $$
\end{thmimp}

\begin{proof}
  Il fatto che $|f| \in R(a, b)$ sarà dimostrato successivamente. \\
  Si ha che:

  $$
    -|f| \leqslant f \leqslant |f|
  $$

  per confronto:

  $$
    \int_{a}^{b}-|f(x)|dx \leqslant \int_{a}^{b}f(x)dx \leqslant \int_{a}^{b}|f(x)|dx
  $$

  per la linearità dell'integrale:

  $$
    -\int_{a}^{b}|f(x)|dx \leqslant \int_{a}^{b}f(x)dx \leqslant \int_{a}^{b}|f(x)|dx
  $$

  quindi:

  $$
    \left|\int_{a}^{b}f(x)dx\right| \leqslant \int_{a}^{b}\left|f(x)\right|dx
  $$
\end{proof}


\begin{thm}
  Sia $f: [a, b] \to \mathbb{R}$ limitata, allora:

  $$
    f \in R(a, b) \iff \forall \varepsilon > 0~\exists~\text{ due costanti a tratti }~h^-,~h^+~\colon~h^- \leqslant f \leqslant h^+~\wedge~\int_{a}^{b}h^+ - h^- < \varepsilon
  $$
\end{thm}

\begin{proof}\leavevmode
  \begin{itemize}
    \item ($\Rightarrow$) L'implicazione diretta è banale e segue dal fatto che $\omega_n(f) = S_n - \sigma_n \to 0$
    \item ($\Leftarrow$) Si vuole mostrare che $\omega_n(f) \to 0$. Si fissi $\varepsilon > 0$ e siano $h^-$, $h^+$ t.che $h^- \leqslant f \leqslant h^+$, con $\int_{a}^{b}h^+ - h^- < \varepsilon$. Allora:
          \begin{align*}
            0 \leqslant \omega_n(f) = S_n - \sigma_n  &\leqslant S_n(h^+) - \sigma_n(h^-) = \\
                                                      &= S_n(h^+) - \int_{a}^{b}h^+ + \int_{a}^{b}h^+ - \int_{a}^{b}h^- + \int_{a}^{b}h^- - \sigma_n(h^-) \leqslant \\
                                                      &\leqslant \left|S_n(h^+) - \int_{a}^{b}h^+\right| + \int_{a}^{b}h^+ - h^- + \left|\int_{a}^{b}h^- - \sigma_n(h^-)\right| < 3\varepsilon
          \end{align*}
  \end{itemize}
\end{proof}


\begin{thmimp}[Additività dell'integrale]
  Sia dato $[a, b]$ e $c \in (a, b)$, allora:

  $$
    f \in R(a, b) \iff f \in R(a, c) \wedge f \in R(c, b)
  $$
  $$
    \int_{a}^{b}f(x)dx = \int_{a}^{c}f(x)dx + \int_{c}^{b}f(x)dx
  $$
\end{thmimp}

\begin{proof}

\end{proof}


\begin{prop}
  Sia $f \in R(-a, a)$:

  $$
    \int_{-a}^{0}f(-x)dx = \int_{0}^{a}f(x)dx
  $$
\end{prop}

\begin{proof}

\end{proof}


\begin{prop}
  Sia $f \in R(-a, a)$ una funzione pari. Allora:

  $$
    \int_{-a}^{a}f(x)dx = 2\int_{0}^{a}f(x)dx
  $$
\end{prop}

\begin{prop}
  Sia $f \in R(-a, a)$ una funzione dispari. Allora:

  $$
    \int_{-a}^{a}f(x)dx = 0
  $$
\end{prop}


\begin{thm}
  Sia $f:[a, b] \to \mathbb{R}$, se $f$ è limitata in $[a, b]$ e $f \in R(c, d)~\forall a < c < d < b$, allora:

  $$
    f \in R(a, b)
  $$

  $$
    \int_{a}^{b}f(x)dx = \lim_{c\to a~d\to b}\int_{c}^{d}f(x)dx
  $$
\end{thm}


\section{Classi di funzioni integrabili}

\begin{thm}[Integrabilità delle continue]
  Sia $f:[a, b] \to \mathbb{R}$. Se $f$ è continua $\implies$ $f \in R(a, b)$.
\end{thm}

\begin{proof}
  Per il teorema di Heine-Cantor, $f$ è uniformemente continua su $[a, b]$. Quindi:

  $$
    \forall \varepsilon > 0~\exists \delta~\colon~\text{ se }~|x - y| < \delta \implies |f(x) - f(y)| < \varepsilon~\forall x,y \in [a, b]
  $$

  Si fissi $\varepsilon > 0$, sia $n_0 \in \mathbb{N}_0$ t.che $\dfrac{b-a}{n} < \delta~\forall n \geqslant n_0$. \\
  Si suddivida $[a, b]$ in $n$ intervalli $\mathfrak{I}_k$ di lunghezza $\dfrac{b-a}{n} < \delta$. Si ha che:

  $$
    \forall x,y \in \mathfrak{I}_k~~~|f(x) - f(y)| < \varepsilon \implies \sup_{\mathfrak{I}_k}f - \inf_{\mathfrak{I}_k} < \varepsilon
  $$

  quindi:

  $$
    \omega_n(f) = \dfrac{b-a}{n}\sum_{k=1}^{n}\sup_{\mathfrak{I}_k} f - \inf_{\mathfrak{I}_k}f < \dfrac{b-a}{n}\sum_{k=1}^{n}\varepsilon = (b-a)\varepsilon
  $$

  di conseguenza $\omega_n(f) \to 0$.
\end{proof}


\begin{thm}
  Sia $f:[a, b] \to \mathbb{R}$ limitata e con al più un numero finito di 
  punti di discontinuità $\implies$ $f \in R(a, b)$.
\end{thm}

\begin{proof}
  Sia $f$ discontinua in $c \in (a, b)$. 
  Per l'additività, se $f \in R(a, c)$ e $f \in R(c, d)$ $\implies$ $f \in R(a, b)$. \\

  Siano $a < \alpha < \beta < c$. $f$ è limitata in $[a, c]$ e continua in $(a, c)$. \\
  $f \in C([\alpha, \beta]) \implies f \in R(\alpha, \beta)$. $f$ è limitata in $[a, c]$ e 
  $f \in R(\alpha, \beta)~\forall a < \alpha < \beta < c$. \\
  Quindi $f \in R(a, c)$.
\end{proof}


\begin{thm}[Integrabilità delle monotone]
  Sia $f: [a, b] \to \mathbb{R}$ monotona $\implies$ $f \in R(a, b)$.
\end{thm}

\begin{proof}
  Si consideri il caso di $f$ crescente e si supponga $[a, b] = [0, 1]$ (la dimostrazione può 
  tuttavia essere estesa a un qualsiasi intervallo). Si ha che, fissato $n \in \mathbb{N}$:

  $$
    S_n(f) = \dfrac{1}{n}\left[\sup_{\mathfrak{I}_1}f + \sup_{\mathfrak{I}_2}f + \dots + \sup_{\mathfrak{I}_n}f\right] = \dfrac{1}{n}\left[f\left(\dfrac{1}{n}\right) + f\left(\dfrac{2}{n}\right) + \dots + f\left(\dfrac{n}{n}\right)\right]
  $$
  $$
    \sigma_n(f) = \dfrac{1}{n}\left[\inf_{\mathfrak{I}_1}f + \inf_{\mathfrak{I}_2}f + \dots + \inf_{\mathfrak{I}_n}f\right] = \dfrac{1}{n}\left[f\left(\dfrac{0}{n}\right) + f\left(\dfrac{1}{n}\right) + \dots + f\left(\dfrac{n-1}{n}\right)\right]
  $$

  dove nel secondo passaggio si è usata la crescenza della funzione. Quindi:

  $$
    \omega_n(f) = S_n(f) - \sigma_n(f) = \dfrac{1}{n}[f(1) - f(0)] \to 0
  $$
\end{proof}


\begin{thm}
  Sia $f \in R(a, b)$ e $\varphi: [c, d] \to \mathbb{R}$ una funzione continua t.che:

  $$
    [a, b] \stackrel{f}{\curvearrowright}\mathfrak{I} \subset [c, d] \stackrel{\varphi}{\curvearrowright} \mathbb{R}
  $$

  $\implies$ $(\varphi \circ f)(x) = \varphi(f(x)) \in R(a, b)$
\end{thm}

\begin{obs}
  Una composta di due integrabili può non essere integrabile.
\end{obs}

\begin{cor}
  Se $f \in R(a, b)$, allora $|f| \in R(a, b)$ e $f^2 \in R(a, b)$
\end{cor}

\begin{cor}
  Se $f, g \in R(a, b)$ $\implies$ $(f \cdot g) \in R(a, b)$
\end{cor}

\begin{proof}
  Si ha che:

  $$
    (f + g) \in R(a, b)
  $$
  $$
    (f - g) \in R(a, b)
  $$

  quindi:

  $$
    (f + g)^2 \in R(a, b)
  $$
  $$
    (f - g)^2 \in R(a, b)
  $$

  di conseguenza:

  $$
    \dfrac{1}{4}\left[(f + g)^2 - (f - g)^2\right] \in R(a, b)
  $$
  $$
    \dfrac{1}{4}\left[(f + g)^2 - (f - g)^2\right] = fg
  $$
\end{proof}


\section{Integrale definito}

\begin{define}
  Sia $f \in R(a, b)$, $x, y \in [a, b]$, si definisce \textbf{integrale definito} tra 
  $x$ e $y$:

  $$
    \int_{x}^{y}f(t)dt =  \begin{cases*}
                            \displaystyle \int_{x}^{y}f(t)dt~\text{ se }~x < y \\
                            0~\text{ se }~x = y \\
                            \displaystyle -\int_{y}^{x}f(t)dt~\text{ se }~x > y
                          \end{cases*}
  $$
\end{define}

\begin{notation}
  $\displaystyle \int_{x}^{y}f(t)dt$, $\displaystyle \int_{[x, y]}f(t)dt$
\end{notation}


\begin{property}[Linearità]

\end{property}

\begin{property}[Additività]
  
\end{property}

\begin{prop}[Confronto]
  
\end{prop}


\section{Primitiva}

\begin{defineimp}
  Sia $f: [a, b] \to \mathbb{R}$, si dice \textbf{primitiva} di $f$ 
  una funzione derivabile $F: [a, b] \to \mathbb{R}$ t.che:

  $$
    F'(x) = f(x)~\forall x \in [a, b]
  $$
\end{defineimp}

\begin{prop}
  Data una funzione $f$, non è detto che esista una primitiva di $F$ di $f$.
\end{prop}

\begin{proof}
  È sufficiente scegliere una $f$ che non abbia la proprietà di Darboux.
\end{proof}


\begin{prop}
  Si supponga che una funzione $f$ ammetta come primitiva $F$. Allora 
  ogni funzione $F(x) + C$ ($C \in \mathbb{R}$) è primitiva di $F$.
\end{prop}

\begin{proof}
  Sia $G(x) = F(x) + C$. Si ha che:

  $$
    G'(x) = F'(x) + \dfrac{d}{dx}C = F'(x) = f(x)
  $$
\end{proof}

\begin{prop}
  Si supponga che una funzione $f$ ammetta come primitiva $F$ e si supponga che $G$ sia anch'essa
  una primitiva di $f$. Allora:

  $$
    G(x) = F(x) + C~~~(C \in \mathbb{R})
  $$
\end{prop}

\begin{proof}
  Sia $H(x) := G(x) - F(x)$, $H$ è derivabile (dato che $F$ e $G$ sono derivabili) e:

  $$
    H'(x) = G'(x) - F'(x) = f(x) - f(x) = 0~\forall x
  $$

  quindi $H(x)$ è una funzione costante.
\end{proof}


\begin{prop}
  Sia $f: [a, b] \to \mathbb{R}$, $f$ ammette primitiva in $[a, b]$ $\centernot \iff$ 
  $f \in R(a, b)$.
\end{prop}

\begin{es}[$\centernot \Leftarrow$]
  È sufficiente prendere una $f$ costante a tratti.
\end{es}

\begin{es}[$\centernot \Rightarrow$]

\end{es}


\begin{prop}
  Se $f$ è limitata e $f$ ammette primitiva $\centernot \implies$ $f \in R(a, b)$.
\end{prop}

\begin{notation}
  $f \in R_P(a, b)$ se:

  \begin{enumerate}[I]
    \item $f$ ammette primitiva in $[a, b]$
    \item $f \in R(a, b)$
  \end{enumerate}
\end{notation}


\section{Teorema fondamentale del calcolo integrale}

\begin{thmimp}[Teorema fondamentale del calcolo integrale]
  Sia $f \in R_P(a, b)$ e sia $F$ una primitiva di $f$, allora:

  $$
    \int_{a}^{b}f(x)dx = F(b) - F(a)
  $$
\end{thmimp}

\begin{notation}
  $F\vert_{a}^{b} := F(b) - F(a)$
\end{notation}

\begin{proof}
  Si dimostrerà il caso $[a, b] = [0, 1]$ (la dimostrazione può tuttavia essere estesa al caso 
  generale). Si fissi $n \in \mathbb{N}$ e si consideri l'intervallo $\mathfrak{I}_k = [x_{k-1}, x_k]$. Su tale 
  intervallo $F$ soddisfa le ipotesi del teorema di Lagrange. Quindi:

  $$
    \exists y \in [x_{k-1}, x_k]~\colon~F(x_{k-1}) - F(x_k) = F'(y)(x_{k-1} - x_k) = \dfrac{1}{n}f(y)
  $$

  dato che $y \in [x_{k-1}, x_k]$:

  $$
    \inf_{\mathfrak{I}_k}f \leqslant y \leqslant \sup_{\mathfrak{I}_k}f~~~\forall k = 1,2,\dots,n
  $$
  $$
    \dfrac{1}{n}\inf_{\mathfrak{I}_k}f \leqslant F(x_{k-1}) - F(x_k) \leqslant \dfrac{1}{n}\sup_{\mathfrak{I}_k}f~~~\forall k = 1,2,\dots,n
  $$

  sommando le $n$ disuguaglianze:

  $$
    \dfrac{1}{n}\sum_{k=1}^{n}\inf_{\mathfrak{I}_k}f \leqslant \sum_{k=1}^{n}F(x_{k-1}) - F(x_k) \leqslant \dfrac{1}{n}\sum_{k=1}^{n}\sup_{\mathfrak{I}_k}f
  $$

  il termine centrale corrisponde alla succesione delle somme parziali di una serie telescopica. Quindi:

  $$
    \sigma_n \leqslant F(1) - F(0) \leqslant S_n~~~\forall n
  $$

  dato che $f \in R(a, b)$, $\sigma_n \to \sigma$, $S_n \to S$, $S = \sigma$. Quindi, per confronto:

  $$
    \int_{0}^{1}f(x)dx = F(1) - F(0)
  $$
\end{proof}


\section{Media integrale}


\begin{define}[Media integrale]
  Data $f \in R(a, b)$, si definisce \textbf{media integrale} di $f$ su $[a, b]$:

  $$
    M_{[a, b]} := \dfrac{1}{b-a}\int_{a}^{b}f(x)dx
  $$
\end{define}

\begin{thm}[Teorema della media integrale]
  Sia $f \in R_P(a, b)$, allora:

  $$
    \exists c \in (a, b)~\colon~\dfrac{1}{b-a}\int_{a}^{b}f(x)dx = f(c)
  $$
\end{thm}

\begin{proof}
  Sia $F$ una primitiva di $f$, si ha che $F$ soddisfa le ipotesi del teorema di Lagrange in $[a, b]$. Quindi:

  $$
    \exists c \in (a, b)~\colon~F'(c) = \dfrac{F(b) - F(a)}{b-a} \implies f(c) = \dfrac{1}{b-a}\int_{a}^{b}f(x)dx
  $$
\end{proof}


\section{Integrazione per parti}

\begin{thmimp}[Integrazione per parti]
  Siano $f, g \in R_P(a, b)$ e siano $F$ e $G$ due primitive rispettivamente di $f$ e $g$, allora:

  $$
    \int_{a}^{b}F(x)g(x)dx = F(x)G(x)\vert_{a}^{b} - \int_{a}^{b}f(x)G(x)dx
  $$
\end{thmimp}

\begin{proof}
  Sia $h(x) = F(x)g(x) + f(x)G(x)$, si ha che $h$ è integrabile ($F$ e $G$ sono continue, in quanto derivabili), inoltre:

  $$
    [F(x)G(x)]' = h(x)
  $$

  quindi $h \in R_P(a, b)$. Di conseguenza, per il teorema fondamentale del calcolo integrale:

  $$
    \int_{a}^{b}h(x)dx = F(x)G(x)\vert_{a}^{b} \implies F(x)G(x)\vert_{a}^{b} = \int_{a}^{b}F(x)g(x)dx + \int_{a}^{b}f(x)G(x)dx
  $$
\end{proof}


\section{Funzione integrale}

\begin{defineimp}
  Sia $f \in R(a, b)$, si definisce \textbf{funzione integrale} di $f$ la funzione $\mathcal{I}: [a, b] \to \mathbb{R}$ così definita:

  $$
    \mathcal{I}(x) = \int_{a}^{x}f(t)dt
  $$
\end{defineimp}

\begin{obs}
  Sia $f \in R(a, b)$, se $a < x < b$, allora $f \in R(a, x)$.
\end{obs}

\begin{obs}
  Si noti che la funzione è ben definita grazie all'integrabilità di $f$.
\end{obs}


\begin{thm}\leavevmode
  \begin{enumerate}[I]
    \item Se $f \in R_P(a, b)$ $\implies$ $\mathcal{I}$ è una primitiva di $f$
    \item La funzione $\mathcal{I}$ è (Lipschitz) continua
    \item Se $f$ è continua in $x_0 \in [a, b]$ $\implies$ $\mathcal{I}$ è derivabile in $x_0$ e:
          $$
            \mathcal{I}'(x_0) = f(x_0)
          $$
  \end{enumerate}
\end{thm}

\begin{proof}[Dimostrazione I]
  Sia $F$ una primitiva di $f$, allora, per il teorema fondamentale del calcolo integrale:

  $$
    \mathcal{I}(x) = F(x) - F(a)
  $$
  
  quindi:

  $$
    \mathcal{I}'(x) = F'(x) = f(x)
  $$
\end{proof}


\begin{proof}[Dimostrazione II]
  Si ha che, dato che $f \in R(a, b)$, $f$ è limitata, quindi $\exists L \geqslant 0$ t.che $|f(x)| \leqslant L~~~\forall x \in [a, b]$. Inoltre:

  \begin{align*}
    |\mathcal{I}(x) - \mathcal{I}(y)| &= \left|\int_{a}^{x}f(t)dt - \int_{a}^{y}f(t)dt\right| = \\
                                      &= \left|\int_{x}^{y}f(t)dt\right| \leqslant \int_{x}^{y}|f(t)|dt \\
                                      &\leqslant \int_{x}^{y}L dt = L(y - x)
  \end{align*}
\end{proof}


\begin{proof}[Dimostrazione III]
  Si dimostrerà che $I'_+(x_0) = f(x_0)$. Dato che $f$ è continua in $x_0$, fissato $\varepsilon > 0$:

  $$
    \exists \delta > 0~\colon~\text{ se }~|x - x_0| < \delta~\implies~|f(x) - f(x_0)| < \varepsilon
  $$

  dunque, sia $h > 0$:

  \begin{align*}
    \left|\dfrac{\mathcal{I}(x_0 + h) - \mathcal{I}(x_0)}{h} - f(x_0)\right|  &= \left|\dfrac{\displaystyle \int_{a}^{x_0 + h}f(t)dt - \int_{a}^{x_0}f(t)dt}{h} - f(x_0)\right| = \\
                                                                              &= \dfrac{1}{h}\left|\int_{x_0}^{x_0 + h}f(t)dt - hf(x_0)\right|
  \end{align*}

  dove nel secondo passaggio si è sfruttata l'additività dell'integrale definito. Si ha inoltre:

  \begin{align*}
    \dfrac{1}{h}\left|\int_{x_0}^{x_0 + h}f(t)dt - hf(x_0)\right| &= \dfrac{1}{h}\left|\int_{x_0}^{x_0 + h}f(t)dt - \int_{x_0}^{x_0 + h}f(x_0)dt\right| = \\
                                                                  &= \dfrac{1}{h}\left|\int_{x_0}^{x_0 + h}f(t) - f(x_0)dt\right| \leqslant \\
                                                                  &\leqslant \dfrac{1}{h}\int_{x_0}^{x_0 + h}|f(t) - f(x_0)|dt < \\
                                                                  &< \dfrac{1}{h}\int_{x_0}^{x_0 + h} \varepsilon dt = \varepsilon
  \end{align*}

  dove il penultimo passaggio è vincolato alla condizione $|x - x_0| < \delta$.
\end{proof}


\begin{cor}
  Se $f$ è continua in $[a, b]$, $\mathcal{I}(x)$ è derivabile in ogni $x \in [a, b]$:

  $$
    \mathcal{I}'(x) = f(x)~~~\forall x \in [a, b]
  $$

  quindi $\mathcal{I}$ è una primitiva di $f$, quindi $f \in R_P(a, b)$.
\end{cor}


\begin{prop}[Formula del trasporto]
  Sia $f \in R_P(a, b)$ e siano $g, h: [\alpha, \beta] \to [a, b]$. Sia $\mathcal{J}: [\alpha, \beta] \to \mathbb{R}$:

  $$
    \mathcal{J}(x) = \int_{h(x)}^{g(x)}f(t)dt
  $$

  sia $F$ una primitiva di $f$, allora:

  $$
    \mathcal{J}(x) = F(g(x)) - F(h(x))
  $$

  se inoltre $g$ e $h$ sono derivabili, allora:

  $$
    \mathcal{J}'(x) = F'(g(x))g'(x) - F'(h(x))h'(x) = f(g(x))g'(x) - f(h(x))h'(x)
  $$
\end{prop}


\section{Cambio di variabili}

\begin{thm}
  Sia $f: [a, b] \to \mathbb{R}$ una funzione continua e sia 
  $\psi: [\alpha, \beta] \to [a, b]$ una funzione derivabile, con $\psi' \in R(a, b)$, allora:

  $$
    \int_{\psi(\alpha)}^{\psi(\beta)}f(x)dx = \int_{\alpha}^{\beta}f(\psi(t))\psi'(t)dt
  $$
\end{thm}

\begin{proof}
  Si ha che $f(\psi(t))\psi'(t) \in R(\alpha, \beta)$ (integrabilità della composta, con $f$ continua).

  Sia $F$ una primitiva di $f$. $F(\psi(t))$ è primitiva di $f(\psi(t))\psi'(t)$, quindi:

  $$
    f(\psi(t))\psi'(t) \in R_P(\alpha, \beta)
  $$

  inoltre:

  $$
    \int_{\alpha}^{\beta}f(\psi(t))\psi'(t)dt = F(\psi(\beta)) - F(\psi(\alpha)) = \int_{\psi(\alpha)}^{\psi(\beta)}f(x)dx
  $$
\end{proof}


\begin{cor}[Cambio di variabili]
  Sia $\psi: [\alpha, \beta] \to [a, b]$ una funzione biunivoca (quindi monotona)

  \begin{itemize}
    \item Se $\psi$ è crescente $\implies$ $\psi(\alpha) = a$, $\psi(\beta) = b$, $\int_{a}^{b}f(x)dx = \int_{\alpha}^{\beta}f(\psi(t))\psi'(t)dt$
    \item Se $\psi$ è decrescente $\implies$ $\psi(\alpha) = b$, $\psi(\beta) = a$, $\int_{a}^{b}f(x)dx = \int_{\beta}^{\alpha}f(\psi(t))\psi'(t)dt$
  \end{itemize}
\end{cor}


\begin{notation}[Differenziale]
  $d(\psi(t)) = \psi'(t)dt$
\end{notation}


\section{Integrale indefinito}

\begin{defineimp}[Integrale indefinito]
  Sia $f$ una funzione che ammetta primitiva in $[a, b]$, si definisce \textbf{integrale indefinito} di 
  $f$ in $[a, b]$ l'insieme delle primitive di $f$.
\end{defineimp}

\begin{notation}[Integrale indefinito]
  $\displaystyle \int f(x)dx$
\end{notation}

\begin{obs}
  Se $F$ è una primitiva di $f$, allora:

  \begin{align*}
    \int f(x)dx &= F(x) + C~~~(C \in \mathbb{R}) \\
                &= \bigcup_{C \in \mathbb{R}} F(x) + C 
  \end{align*}
\end{obs}


\begin{property}[Linearità]
  $$
    \int f(x) + \lambda g(x) dx = \int f(x) dx + \lambda \int g(x) dx
  $$
\end{property}

\begin{property}[Integrazione per parti]
  $$
    \int f(x)g'(x) dx = f(x)g(x) - \int f'(x)g(x)dx
  $$
\end{property}

\begin{property}[Cambio di variabili]
  $$
    \int f(\psi(t))\psi'(t)dt = F(\psi(t)) + C~~~C \in \mathbb{R}
  $$

  se $\psi$ è invertibile:

  $$
    \int f(x)dx = \int f(\psi(t))\psi'(t)dt|_{t=\psi^{-1}(x)}
  $$
\end{property}


\section{Applicazione degli integrali allo studio delle serie numeriche}

\begin{thm}
  Se $\alpha > 1$, allora:

  $$
    \sum_{n=1}^{\infty}\dfrac{1}{n^{\alpha}}~\conv
  $$

  inoltre:

  $$
    \zeta(\alpha) < \dfrac{\alpha}{\alpha - 1}
  $$
\end{thm}

\begin{proof}\leavevmode

  \begin{center}
    \begin{tikzpicture}
    \begin{axis}[
      width=8cm,
      height=8cm,
      xmin=-0.7, xmax=7.5,
      ymin=-0.2, ymax=1.2,
      axis lines=middle,
      axis line style={->},
      xtick=\empty,
      ytick=\empty,
    ]

    \addplot[domain=1:7, smooth, very thick, tyrian]
        {1/x^3};

    \addplot[
      ybar interval,
      fill=cmapteal-300,
      fill opacity=0.35,
      draw=cmapteal-300,
    ] coordinates {
      (0,1)
      (1,{1/2^3})
      (2,{1/3^3})
      (3,{1/4^3})
      (4,{1/5^3})
      (5,0)
    };

    % Labels
    \node[left] at (axis cs:0,1) {$1$};
    \node[left] at (axis cs:0,{1/2^3+0.1}) {$\frac{1}{2^{\alpha}}$};
    \node[left] at (axis cs:0,{1/3^3+0.05}) {$\frac{1}{3^{\alpha}}$};
    \node[below left] at (axis cs:0,0) {$\frac{1}{n^{\alpha}}$};

    \node[below] at (axis cs:1,0) {$1$};
    \node[below] at (axis cs:2,0) {$2$};
    \node[below] at (axis cs:3,0) {$3$};
    \node[below] at (axis cs:6,0) {$n$};

    \node[above] at (axis cs:0.5,1) {$\color{cmapteal-300} h(x)$};
    \node[above] at (axis cs:2,{1/2^3+0.2}) {$\color{tyrian} f(x)$};

    \end{axis}
    \end{tikzpicture}
  \end{center}

  Sia $s_n := \sum_{k=1}^{n} \dfrac{1}{k^{\alpha}}$ e $f(x) := \dfrac{1}{x^{\alpha}}$. Si ha che:

  $$
    s_n = \int_{0}^{n}h(x)dx = 1 + \int_{1}^{n}h(x)dx
  $$

  inoltre, su $[0, +\infty)$, si ha che $h(x) < f(x)$. Quindi, per il teorema sul confronto degli integrali definiti:

  $$
    1 + \int_{1}^{n}h(x)dx \leqslant 1 + \int_{1}^{n}\dfrac{1}{x^{\alpha}}dx
  $$

  quindi:

  \begin{align*}
    s_n \leqslant 1 + \left[\dfrac{x^{-\alpha + 1}}{-\alpha + 1}\right]^n_1 &= 1 + \left(\dfrac{n^{-\alpha + 1}}{1 - \alpha} - \dfrac{1}{1 - \alpha}\right) \\
                                                                            &= 1 + \dfrac{1}{\alpha - 1}\left(1 - n^{1 - \alpha}\right) = \\
                                                                            &= \dfrac{\alpha}{\alpha - 1} - \dfrac{n^{1 - \alpha}}{\alpha - 1} < \dfrac{\alpha}{\alpha - 1}
  \end{align*}


\end{proof}


\begin{thm}
  $$
    \sum_{n=0}^{\infty}\dfrac{(-1)^n}{2n+1} = \dfrac{\pi}{4}
  $$
\end{thm}

\begin{proof}
  Sia $s_n := \sum_{k=0}^{n}\dfrac{(-1)^k}{2k+1}$. Si noti che:

  $$
    \int_{0}^{1}x^{2k} dx = \dfrac{1}{2k + 1}
  $$

  quindi:

  $$
    s_n = \sum_{k=0}^{n}(-1)^k \int_{0}^{1}x^{2k}dx
  $$

  per la linearità dell'integrale:

  $$
    s_n = \int_{0}^{1}\sum_{k=0}^{n}(-x^2)^k dx
  $$

  l'integranda è ora la successione delle somme parziali di una serie geometrica, la cui somma risulta:

  $$
    \sum_{k=0}^{n}(-x^2)^k = \dfrac{(-x^2)^{n+1} - 1}{-x^2 - 1}
  $$

  di conseguenza:

  $$
    s_n = \int_{0}^{1} \dfrac{(-x^2)^{n+1} - 1}{-x^2 - 1} dx = \int_{0}^{1} \dfrac{1}{x^2 + 1}dx + R_n = \dfrac{\pi}{4} + R_n
  $$

  dove:

  $$
    R_n := -\int_{0}^{1} \dfrac{(-x^2)^{n+1}}{x^2 + 1} dx
  $$

  si vuole quindi mostrare che $R_n \to 0$. Si ha quindi:

  \begin{align*}
    |R_n| = \left|\int_{0}^{1} \dfrac{(-x^2)^{n+1}}{x^2 + 1} dx\right|  &\leqslant \int_{0}^{1}\left|\dfrac{(-x^2)^{n+1}}{x^2 + 1}\right| dx = \\
                                                                        &= \int_{0}^{1}\dfrac{x^{2n+2}}{x^2 + 1} dx \leqslant \int_{0}^{1}x^{2n + 2}dx = \\
                                                                        &= \dfrac{1}{2n + 3} \to 0
  \end{align*}
\end{proof}


\begin{thm}[Formula di Eulero-Mascheroni]
  Detta $s_n := \sum_{k=1}^n \dfrac{1}{k}$ si ha che:

  $$
    s_n - \log{n} \to \gamma \in \left(\dfrac{1}{2}, 1\right)
  $$
\end{thm}

\begin{proof}\leavevmode

  \begin{center}
    \begin{tikzpicture}
    \begin{axis}[
      width=8cm,
      height=8cm,
      xmin=-0.7, xmax=7.5,
      ymin=-0.2, ymax=1.2,
      axis lines=middle,
      axis line style={->},
      xtick=\empty,
      ytick=\empty,
    ]

    \addplot[name path=curve, domain=0.99:7, smooth, very thick, tyrian]
        {1/x};

    \addplot[name path=top1, draw=none] {1};

    \addplot[
      cmapteal-500,
      fill opacity=0.35,
      draw=cmapteal-500
    ]
    fill between[
      of=top1 and curve,
      soft clip={domain=1:2},
    ];

    \addplot[name path=top2, draw=none] {1/2};

    \addplot[
      cmapteal-500,
      fill opacity=0.35,
      draw=cmapteal-500
    ]
    fill between[
      of=top2 and curve,
      soft clip={domain=2:3},
    ];

    \addplot[name path=top3, draw=none] {1/3};

    \addplot[
      cmapteal-500,
      fill opacity=0.35,
      draw=cmapteal-500
    ]
    fill between[
      of=top3 and curve,
      soft clip={domain=3:4},
    ];

    % Labels
    \node[left] at (axis cs:0,1) {$1$};
    \node[left] at (axis cs:0,{1/2}) {$\frac{1}{2}$};
    \node[left] at (axis cs:0,{1/3}) {$\frac{1}{3}$};
    \node[below left] at (axis cs:0,{1/5}) {$\frac{1}{n}$};

    \node[below] at (axis cs:1,0) {$1$};
    \node[below] at (axis cs:2,0) {$2$};
    \node[below] at (axis cs:3,0) {$3$};
    \node[below] at (axis cs:6,0) {$n$};

    \node[above] at (axis cs:1.5,1) {$\color{cmapteal-500} h(x)$};
    \node[above] at (axis cs:2.5,0.5) {$\color{cmapteal-500} \mathfrak{A}_k$};
    \node[below] at (axis cs:2,{1/2-0.2}) {$\color{tyrian} f(x)$};

    \end{axis}
    \end{tikzpicture}
  \end{center}

  si ha che:

  $$
    \log{(n+1)} = \log{\left[n\left(1 + \dfrac{1}{n}\right)\right]} = \log{n} + \goesto{\log{\left(1 + \dfrac{1}{n}\right)}}{0}
  $$

  di conseguenza, $\log{n}$ e $\log{(n+1)}$ sono uguali a meno di una successione infinitesima. \\
  Detta $S_n$ la successione delle somme parziali:

  $$
    S_n = \int_{1}^{n+1}h(x)dx = \equalto{\int_{1}^{n+1}\dfrac{1}{x}dx}{\log{(n+1)}} + \sum_{k=1}^{n}\mathfrak{A}_k
  $$

  quindi:

  $$
    S_n - \log{(n+1)} = \sum_{k=1}^{n}\mathfrak{A}_k
  $$

  si ha che:

  \begin{center}
  \begin{tikzpicture}
    \begin{groupplot}[
      group style={
        group size=2 by 1,
        horizontal sep=2cm,
      },
      width=8cm,
      height=7cm,
      xmin=0.8, xmax=2.2,
      ymin=0, ymax=1.2,
      axis lines=none,
      ticks=none,
    ]

    % -------- LEFT: curved region --------
    \nextgroupplot

    \addplot[
      fill=tyrian,
      fill opacity=0.05,
      draw=tyrian,
    ]
    coordinates {
      (1,1)
      (2,1)
      (2,0.5)
      (1,0.5)
      (1,1)
      (2,0.5)
    };

    % invisible top boundary y = 1
    \addplot[name path=top, domain=1:2, draw=none] {1};

    % curve y = 1/x
    \addplot[name path=curve, domain=1:2, thick, cmapteal-500]
      {1/x};

    % filled curved region
    \addplot[
      fill=cmapteal-500,
      fill opacity=0.35,
      draw=cmapteal-500
    ]
    fill between[
      of=top and curve,
    ];

    \node[above] at (axis cs:2,1) {$\color{cmapteal-500} \mathfrak{A}_k$};

    % -------- RIGHT: rectangle --------
    \nextgroupplot

    % rectangle with base [1,2] and height 0.5
    \addplot[
      fill=tyrian,
      fill opacity=0.35,
      draw=tyrian,
    ]
    coordinates {
      (1,1)
      (2,1)
      (2,0.5)
      (1,0.5)
      (1,1)
      (2,0.5)
    };

    \node[above] at (axis cs:2,1) {$\color{tyrian} \mathfrak{A}_{R_k}$};

    \end{groupplot}
  \end{tikzpicture}
  \end{center}

  $$
    \dfrac{1}{2}\mathfrak{A}_{R_k} \leqslant \mathfrak{A}_k \leqslant \mathfrak{A}_{R_k}
  $$

  dove l'altezza del rettangolo è:
  
  $$
    h = \dfrac{1}{k} - \dfrac{1}{k+1}
  $$

  quindi si ha che:

  $$
    \dfrac{1}{2}\sum_{k=1}^{n}\dfrac{1}{k(k+1)} \leqslant \gamma_n \leqslant \sum_{k=1}^{n}\dfrac{1}{k(k+1)}
  $$

  quindi per confronto:

  $$
    \gamma_n \to \gamma \in \left[\dfrac{1}{2}, 1\right]
  $$
\end{proof}


\subsection{Esercizi svolti}

\begin{ex}
  $$
    \int \log{x}dx
  $$

  $$
    \int \log{x}dx = \int 1 \cdot \log{x}dx = x\log{x} - \int \cancel{x} \cdot \dfrac{1}{\cancel{x}}dx = x(\log{x} - 1) + C
  $$
\end{ex}


\begin{ex}
  Determinare una primitiva di una funzione nella forma:

  $$
    f(x) = \dfrac{P(\sin{x}, \cos{x})}{Q(\sin{x}, \cos{x})}
  $$

  dove $P$ e $Q$ sono due polinomi in seno e coseno. \\

  Per determinare una primitiva è sufficiente operare la sostituzione:

  $$
    t := \tan{\dfrac{x}{2}}
  $$

  e sfruttare le formule parametriche di seno e coseno:

  $$
    \sin{x} = \dfrac{2t}{1 + t^2}
  $$
  $$
    \cos{x} = \dfrac{1 - t^2}{1 + t^2}
  $$
\end{ex}


\begin{ex}
  Calcolare:

  $$
    \int \sqrt{a^2 - x^2}dx
  $$
  $$
    \int \sqrt{a^2 + x^2}dx
  $$
  $$
    \int \sqrt{x^2 - a^2}dx
  $$

  È sufficiente operare le seguenti sostituzioni (nell'ordine):

  $$
    \int \sqrt{a^2 - x^2}dx~~~x := a\sin{t}
  $$
  $$
    \int \sqrt{a^2 + x^2}dx~~~x := a\sinh{t}
  $$
  $$
    \int \sqrt{x^2 - a^2}dx~~~x := a\cosh{t}
  $$
\end{ex}


\begin{ex}
  Calcolare:

  $$
    \int \dfrac{1}{\sqrt{x} + \sqrt[3]{x}}dx
  $$

  Si può procedere effettuando la sostituzione:

  $$
    t := x^{\frac{1}{\gcd{(2, 3)}}} = \sqrt[6]{x}
  $$
\end{ex}


\begin{ex}
  Calcolare:

  $$
    I = \int \sin{(\alpha x)}\cos{(\beta x)}dx
  $$

  È sufficiente utilizzare le formule di Prostaferesi:

  $$
    \sin{p} + \sin{q} = 2\sin{\dfrac{p + q}{2}}\cos{\dfrac{p - q}{2}}
  $$
\end{ex}



\section{Integrale improprio (generalizzato)}


\begin{define}
  Sia $f: (a, b) \to \mathbb{R}$ e siano $-\infty \leqslant a < b \leqslant +\infty$, $f$ si 
  dice \textbf{localmente integrabile} in $(a, b)$ se:

  $$
    \forall [c, d] \subset (a, b)~f \in R(c, d)
  $$
\end{define}

\begin{defineimp}[Integrale improprio]
  Sia $f$ localmente integrabile in $(a, b)$, se esiste finito:
  
  $$
    \lim_{c\to a~d\to b}\int_{c}^{d}f(x)dx = l \in \mathbb{R}
  $$

  $f$ si dice \textbf{integrabile impropriamente} su $(a, b)$. Inoltre, $l$ si dice \textbf{integrale improprio} di $f$ 
  su $(a, b)$
\end{defineimp}

\begin{notation}
  $\displaystyle \int_{a}^{b}f(d)dx := l$
\end{notation}

\begin{notation}
  $I(a, b)$, classe delle funzioni integrabili impropriamente su $[a, b]$
\end{notation}


\begin{obs}
  $\displaystyle \int_{a}^{b}f(x)dx = l \in \mathbb{R}$ se:

  $$
    \forall a_n \to a^+~\forall b_n \to b^-~\int_{a_n}^{b_n}f(x)dx \to l
  $$
\end{obs}


\begin{obs}[Additività]

\end{obs}

\begin{obs}[$\diamondsuit$]
  Se $a > -\infty$ e $f$ è limitata in $\mathcal{U}^+(a)$, $\rightarrow$

  $$
    \int_{a}^{b}f(x)dx = \lim_{d \to b}\int_{a}^{d}f(x)dx
  $$
\end{obs}

\begin{prop}
  Sia $\alpha \in \mathbb{R}$, allora:

  $$
    \int_{1}^{+\infty}\dfrac{1}{x^{\alpha}}dx = \begin{cases*}
                                                  \dfrac{1}{\alpha-1}~\text{ se }~\alpha > 1 \\
                                                  +\infty~\text{ se }~\alpha \leqslant 1
                                                \end{cases*}
  $$

  $$
    \int_{0}^{1}\dfrac{1}{x^{\alpha}}dx = \begin{cases*}
                                            \dfrac{1}{\alpha-1}~\text{ se }~\alpha < 1 \\
                                            +\infty~\text{ se }~\alpha \geqslant 1
                                          \end{cases*}
  $$
\end{prop}

\begin{proof}
  È sufficiente calcolare i limiti:

  $$
    \lim_{R \to +\infty}\left[\dfrac{R^{-\alpha + 1}}{-\alpha + 1} - \dfrac{1}{-\alpha + 1}\right] = \lim_{R \to +\infty} \dfrac{1 - R^{-\alpha + 1}}{\alpha - 1}
  $$

  e

  $$
    \lim_{\varepsilon \to 0^+}\left[\dfrac{1}{-\alpha + 1} - \dfrac{\varepsilon^{-\alpha + 1}}{-\alpha + 1}\right] = \lim_{\varepsilon \to 0^+} \dfrac{\varepsilon^{-\alpha + 1} - 1}{\alpha - 1}
  $$
\end{proof}


\begin{define}
  Sia $f: \mathfrak{D}(f) \to \mathbb{R}$, $f$ si dice \textbf{localmente integrabile} su un intervallo $[a ,b]$ se 
  esiste una partizione $a = x_0 < x_1 < \dots < x_k = b$ di $[a, b]$ t.che $f$ è localmente integrabile (nel senso definito in precedenza) su ogni intervallo della forma 
  $[x_{k-1}, x_k]$.
\end{define}


\begin{define}
  Sia $f$ localmente integrabile su $[a, b]$, si definisce \textbf{integrale improprio} su $[a, b]$ (se esiste finito):

  $$
    \int_{a}^{b}f(x)dx := \sum_{k=1}^{n}\int_{x_{k-1}}^{x_k}f(x)dx = l \in \mathbb{R}
  $$
\end{define}

\begin{obs}
  $f \in I(a, b) \iff f \in I(x_{k-1}, x_k)~\forall k=1, 2, \dots, n$
\end{obs}

\begin{obs}
  Sia $\{a_n\}$ una successione e sia:

  $$
    f(x) = a_n \chi_{[n-1, n]}(x)
  $$

  allora:

  $$
    \sum_{n=1}^{\infty}a_n~\conv~\iff~\exists \int_{0}^{+\infty}f(x)dx 
  $$
\end{obs}

\begin{obs}
  In analogia alle serie numeriche, un integrale improprio può convergere (se $f \in I(a, b)$), divergere o essere indeterminato.
\end{obs}


\begin{define}
  $\int_{a}^{b} f$ converge \textbf{assolutamente} se $\int_{a}^{b}|f|$ converge
\end{define}

\begin{obs}
  Se $f \geqslant 0$ su $(a, b)$ $\implies$ $\int_{a}^{b}f(x)dx$ converge a un numero non negativo o diverge a $+\infty$.
\end{obs}


\begin{thm}[Linearità]
  Siano $f, g \in I(a, b)$ e $\lambda \in \mathbb{R}$ $\implies$ $f + \lambda g \in I(a, b)$ e:

  $$
    \int_{a}^{b}f + \lambda g = \int_{a}^{b}f + \lambda \int_{a}^{b}g
  $$
\end{thm}

\begin{proof}
  Per la linearità dell'integrale:

  $$
    \int_{c}^{d}f + \lambda g = \int_{c}^{d}f + \lambda \int_{c}^{d}g
  $$

  $$
    \int_{a}^{b}f + \lambda g = \lim_{c\to a~d\to b}\int_{c}^{d}f + \lambda g = \lim_{c\to a~d\to b}\int_{c}^{d}f + \lambda \int_{c}^{d}g = \int_{a}^{b}f + \lambda \int_{a}^{b}g
  $$
\end{proof}


\begin{thm}[Confronto per l'integrale improprio]
  Siano $0 \leqslant f(x) \leqslant g(x)~\forall x \in (a, b)$, allora:
  \begin{itemize}
    \item se $g \in I(a, b)$ $\implies$ $f \in I(a, b)$
    \item se $f \not \in I(a, b)$ $\implies$ $g \not \in I(a, b)$ 
  \end{itemize}
\end{thm}

\begin{cor}
  Se $\forall x \in (a, b)$ $f, g \geqslant 0$ e $\exists 0 < m < M$ t.che:

  $$
    mf(x) \leqslant g(x) \leqslant Mf(x)~\forall x \in (a, b)
  $$

  allora $f \in I(a, b)~\iff~g \in I(a, b)$
\end{cor}


\begin{thm}[Convergenza assoluta]
  Sia $f$ localmente integrabile, se $|f| \in I(a, b)$ $\implies$ $f \in I(a, b)$
\end{thm}


\begin{obs}
  $f \in I(a, b)$, $g \in I(a, b)$ $\centernot \implies$ $(f\cdot g) \in I(a, b)$
\end{obs}

\begin{es}
  Sia $f(x) = g(x) = \dfrac{1}{\sqrt{x}}$ e $(a, b) = (0, 1)$
\end{es}

\begin{obs}
  Sia $f$ localmente integrabile su $(a, b)$, $f \in I(a, b)$ $\centernot \implies$ $|f| \in I(a, b)$
\end{obs}

\begin{es}

\end{es}


\begin{thmimp}[Confronto asintotico per gli integrali impropri]
  Sia $f$ una funzione di segno costante in $\dot{\mathcal{U}}(x_0)$, allora se $f \sim g$ in $\mathcal{U}(x_0)$ 
  si ha che $\exists \mathcal{U}'(x_0) \subset \mathcal{U}(x_0)$ t.che:

  $$
    f \in I(\mathcal{U}'(x_0)) \iff g \in I(\mathcal{U}'(x_0))
  $$
\end{thmimp}

\begin{proof}
  $$
    \dfrac{f(x)}{g(x)} \underset{x \to x_0}{\to} 1 \implies \text{ in } \mathcal{U}'(x_0)~~\dfrac{1}{2}g(x) \leqslant f(x) \leqslant 2g(x)
  $$
\end{proof}



\begin{thmimp}[Funzione campione]
  Siano $\alpha, \beta \in \mathbb{R}$ e $f$ definita come:

  $$
    f(x) = \dfrac{1}{x^{\alpha}\left|\log x\right|^\beta}~~~x \in (0, +\infty)
  $$

  allora:

  $$
    f \in I(\mathcal{U}(0)) \iff  \begin{cases*}
                                    \alpha < 1~~~\forall \beta \\
                                    \alpha = 1,~\beta > 1
                                  \end{cases*}     
  $$

  $$
    f \in I(\mathcal{U}(+\infty)) \iff  \begin{cases*}
                                          \alpha > 1~~~\forall \beta \\
                                          \alpha = 1,~\beta > 1
                                        \end{cases*}     
  $$
\end{thmimp}

\begin{proof}
  Sia $\alpha > 1$, si può scrivere:

  $$
    \alpha := 1 + 2p
  $$

  con $p > 0$. Quindi, sia $x \in \mathcal{U}(+\infty)$:

  $$
    f(x) = \dfrac{1}{x^{1 + 2p}\left(\log x\right)^\beta} \leqslant \dfrac{1}{x^{1 + p}} \iff \dfrac{x^{1 + p}}{x^{1 + 2p}\left(\log x\right)^\beta} \leqslant 1 \iff x^{p}\left(\log{x}\right)^{\beta} \geqslant 1
  $$

  per confronto, si ha che:

  $$
    \dfrac{1}{x^{\alpha}\left(\log{x}\right)^{\beta}} \leqslant \dfrac{1}{x^{1+p}} \in I(\mathcal{U}(+\infty))
  $$
\end{proof}


\subsection{Esercizi svolti}

\begin{ex}
  Si dimostri che:

  $$
    \lim_{x \to 0^+}\dfrac{1}{x}\int_{0}^{x}\sin{\dfrac{1}{t}}dt = 0
  $$

  Sia $x > 0$ fissato. Si noti che:

  $$
    \left(\cos{\dfrac{1}{t}}\right)' = \sin{\dfrac{1}{t}}\cdot \dfrac{1}{t^2}
  $$

  quindi:
  
  $$
    \int_0^x \sin{\dfrac{1}{t}}dt = \int_0^x t^2\left(\cos{\dfrac{1}{t}}\right)'dt = t^2\cos{\dfrac{1}{t}}\left. \right|_0^x - \int_0^x 2t \cos{\dfrac{1}{t}}dt
  $$

  Il limite diventa dunque:

  $$
    \lim_{x \to 0^+}\left[x\cos{\dfrac{1}{x}} - \dfrac{2}{x}\int_0^x t\cos{\dfrac{1}{t}}dt\right] = \lim_{x \to 0^+}\left[- \dfrac{2}{x}\int_0^x t\cos{\dfrac{1}{t}}dt\right]
  $$

  Inoltre:

  $$
    \dfrac{2}{x}\left|\int_0^x t\cos{\dfrac{1}{t}}dt\right| \leqslant \dfrac{2}{x}\int_0^x t\left|\cos{\dfrac{1}{t}}\right|dt \leqslant \dfrac{2}{x}\int_0^x t dt = x \underset{x \to 0^+}{\to} 0
  $$
\end{ex}


\begin{ex}
  Si dimostri che:

  $$
    f(x) = \dfrac{\sin{x}}{x}
  $$

  è integrabile impropriamente in $(\pi, +\infty)$, ma che non lo è il suo modulo
\end{ex}


\begin{ex}[Integrale di Dirichlet]
  Si calcoli:

  $$
    \int_{0}^{\infty}\dfrac{\sin{x}}{x}dx
  $$
\end{ex}

\begin{ex}
  Si calcoli:

  $$
    \int_{0}^{\pi}\log{(\sin{x})}dx
  $$
\end{ex}


\begin{ex}
  Si calcoli:

  $$
    \int e^{\arcsin{x}}dx
  $$

  $$
    I := \int e^{\arcsin{x}}dx = \int 1\cdot e^{\arcsin{x}}dx = xe^{\arcsin{x}} - \int \dfrac{x}{\sqrt{1 - x^2}}e^{\arcsin{x}}dx
  $$

  $$
    I = xe^{\arcsin{x}} + \int \dfrac{-x}{\sqrt{1 - x^2}}e^{\arcsin{x}}dx = xe^{\arcsin{x}} + \sqrt{1 - x^2}e^{\arcsin{x}} - \int \cancel{\sqrt{1 - x^2}}\dfrac{e^{\arcsin{x}}}{\cancel{\sqrt{1 - x^2}}}dx
  $$

  $$
    I = \dfrac{1}{2}xe^{\arcsin{x}} + \dfrac{1}{2}\sqrt{1 - x^2}e^{\arcsin{x}} + C
  $$
\end{ex}


\begin{ex}
  Si calcoli:

  $$
    \int \dfrac{\sqrt{\log{(\log{x})}}}{x\log{x}}dx
  $$

  si ponga:

  $$
    t := \log{(\log{x})}
  $$

  $$
    dt = \dfrac{1}{\log{x}}\cdot \dfrac{1}{x}dx
  $$

  si ottiene:

  $$
    \int \sqrt{t}dt = \dfrac{2}{3}\left[\log{(\log{x})}\right]^{\frac{2}{3}} + C
  $$
\end{ex}


\begin{ex}[Integrali di Fresnel]
  Si dimostri che:

  $$
    \int_0^{\infty} \sin{(x^2)}dx
  $$

  converge. \\

  Si ponga:

  $$
    t := x^2 \implies x = \sqrt{t}
  $$

  $$
    dx = \dfrac{1}{2\sqrt{t}}dt
  $$

  $$
    I = \int_0^{\infty}\dfrac{1}{2\sqrt{t}}\sin{t}dt = \lim_{R \to \infty}\int_0^{R}\dfrac{\sin{t}}{2\sqrt{t}}dt
  $$

  sia $a > 0$, si ha che:

  $$
    \int_a^R \dfrac{\sin{t}}{2\sqrt{t}}dt = -\dfrac{\cos{t}}{2\sqrt{t}}\left.\right|_a^R - \int_a^R \dfrac{\cos{t}}{t^{\frac{3}{2}}}dt
  $$

  $$
    \int_a^{\infty}\sin{(x^2)}dx = \underset{l \in \mathbb{R}}{\underbrace{\lim_{R \to \infty}\left[-\dfrac{\cos{t}}{2\sqrt{t}}\right]\left. \right|_a^R}} - \int_{a}^{\infty}\dfrac{\cos{t}}{t^{\frac{3}{2}}}dt
  $$

  ma, dato che il coseno è limitato, per confronto:

  $$
    \int_{a}^{\infty}\left|\dfrac{\cos{t}}{t^{\frac{3}{2}}}\right|dt~~~\conv
  $$

  di conseguenza:

  $$
    \int_a^{\infty}\sin{(x^2)}dx~~~\conv
  $$
\end{ex}


\begin{ex}
  Si calcoli:

  $$
    \int \dfrac{1}{(1 + x^2)^3}dx
  $$

  È sufficiente effettuare la sostituzione:

  $$
    x := \tan{u}
  $$
\end{ex}


\begin{ex}
  Si calcoli:

  $$
    I := \int_{-1}^{1}\dfrac{1}{(1 + x^2)(1 + e^x)}dx
  $$

  È sufficiente effettuare la sostituzione:

  $$
    u := -x
  $$
\end{ex}