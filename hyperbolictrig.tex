\chapter{Trigonometria iperbolica}

\begin{define}[Coseno iperbolico]
  Si definisce \textbf{coseno iperbolico} la funzione $\cosh : \mathbb{R} \to \mathbb{R}$:

  $$
    \cos{x} := \dfrac{e^x + e^{-x}}{2}
  $$
\end{define}

\begin{define}[Seno iperbolico]
  Si definisce \textbf{seno iperbolico} la funzione $\sinh : \mathbb{R} \to \mathbb{R}$:

  $$
    \sinh{x} := \dfrac{e^x - e^{-x}}{2}
  $$
\end{define}

\begin{define}[Tangente iperbolica]
  Si definisce \textbf{tangente iperbolica} la funzione $\tanh : \mathbb{R} \to \mathbb{R}$:

  $$
    \tanh{x} := \dfrac{\sinh{x}}{\cosh{x}} = \dfrac{e^x - e^{-x}}{e^{x} + e^{-x}}
  $$
\end{define}

\begin{define}[Cotangente iperbolica]
  Si definisce \textbf{cotangente iperbolica} la funzione $\tanh : \mathbb{R}_0 \to \mathbb{R}$:

  $$
    \coth{x} := \dfrac{\cosh{x}}{\sinh{x}} = \dfrac{e^x + e^{-x}}{e^{x} - e^{-x}}
  $$
\end{define}

\begin{center}
\begin{tikzpicture}
\begin{groupplot}[
  group style={
    group size=2 by 1,
    horizontal sep=2.5cm,
    ylabels at=edge left,
    xlabels at=edge bottom,
    vertical sep=0pt
  },
  width=8.25cm,
  height=8.5cm,
  axis lines=middle,
  xlabel={$x$},
  samples=300,
  domain=-3:3,
  ticks=none,
  ymin=-3, ymax=6,
  legend style={font=\small, cells={anchor=west}},
  every axis title/.append style={yshift=-1ex}
]

\nextgroupplot[
  xmin=-2.5, xmax=2.5,
  ylabel={$y$},
  ylabel style={xshift=2.5pt}
]
\addplot[cmapteal-200, very thick] {sinh(x)};
\addlegendentry{$\sinh(x)$}
\addplot[cmapteal-400, very thick] {cosh(x)};
\addlegendentry{$\cosh(x)$}

\nextgroupplot[
  xmin=-3, xmax=3,
  ylabel={$y$}, 
  ylabel style={xshift=2.5pt}
]
\addplot[cmapteal-200, very thick] {tanh(x)};
\addlegendentry{$\tanh(x)$}
\addplot[cmapteal-400, very thick, domain=-3:-0.2] {cosh(x)/sinh(x)};
\addplot[cmapteal-400, very thick, domain=0.2:3] {cosh(x)/sinh(x)};
\addlegendentry{$\coth(x)$}

\end{groupplot}
\end{tikzpicture}
\end{center}


\begin{prop}\leavevmode
  $$
    \cosh^2{x} - \sinh^2{x} = 1~\forall x \in \mathbb{R}
  $$
\end{prop}

\begin{prop}
  Siano $\alpha, \beta \in \mathbb{R}$:

  \begin{itemize}
    \item $\sinh{(\alpha + \beta)} = \sinh{\alpha}\cosh{\beta} + \sinh{\beta}\cosh{\alpha}$
    \item $\cosh{(\alpha + \beta)} = \cosh{\alpha}\cosh{\beta} + \sinh{\alpha}\sinh{\beta}$
  \end{itemize}
\end{prop}

\begin{proof}

\end{proof}

\begin{cor}
  Siano $\alpha, \beta \in \mathbb{R}$:

  \begin{itemize}
    \item $\sinh{(\alpha - \beta)} = \sinh{\alpha}\cosh{\beta} - \sinh{\beta}\cosh{\alpha}$
    \item $\cosh{(\alpha - \beta)} = \cosh{\alpha}\cosh{\beta} - \sinh{\alpha}\sinh{\beta}$
  \end{itemize}
\end{cor}

\begin{cor}
  Sia $\alpha \in \mathbb{R}$:

  \begin{itemize}
    \item $\sinh{(2\alpha)} = 2\sinh{\alpha}\cosh{\alpha}$
    \item $\cosh{(2\alpha)} = \cosh^2{\alpha} + \sinh^2{\alpha}$
  \end{itemize}
\end{cor}

\begin{prop}
  Sia $\alpha \in \mathbb{R}$:

  \begin{itemize}
    \item $\sinh{\left(\dfrac{\alpha}{2}\right)} = \sgn{(\alpha)}\sqrt{\dfrac{\cosh{\alpha} - 1}{2}}$
    \item $\cosh{\left(\dfrac{\alpha}{2}\right)} = \sqrt{\dfrac{\cosh{\alpha} + 1}{2}}$
  \end{itemize}
\end{prop}

\begin{proof}

\end{proof}


\begin{prop}
  La funzione $\sinh : \mathbb{R} \to \mathbb{R}$ è invertibile e l'inversa è data da:
\end{prop}