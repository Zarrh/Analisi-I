\chapter{Funzioni Reali}

\section{Topologia}

\begin{define}
  Sia $x \in \overline{\mathbb{R}}$, si definisce \textbf{intorno} 
  $\mathcal{U}(x)$ di $x$:

  $$
    \mathcal{U}(x) = 
    \begin{cases}
      (x - \varepsilon, x + \varepsilon)~(\varepsilon > 0)~\text{se}~x \in \mathbb{R} \\
      (M, +\infty)~(M > 0)~\text{se}~x = +\infty \\
      (-\infty, -M)~(M > 0)~\text{se}~x = -\infty
    \end{cases}
  $$
\end{define}

\begin{define}
  Sia $x \in \overline{\mathbb{R}}$, si definisce \textbf{intorno bucato} 
  $\dot{\mathcal{U}}(x)$ di $x$:

  $$
    \dot{\mathcal{U}}(x) = 
    \begin{cases}
      \mathcal{U}(x)\setminus \{x\}~\text{se}~x \in \mathbb{R} \\
      \mathcal{U}(x)~\text{se}~x = \pm \infty \\
    \end{cases}
  $$
\end{define}


\begin{obs}
  $x_n \to x \iff~\forall \mathcal{U}(x)$ si ha che $x_n \in \mathcal{U}(x)~\definit$
\end{obs}


\begin{defineimp}[Punto di accumulazione]
  Un punto $x \in \overline{\mathbb{R}}$ è detto \textbf{punto di accumulazione} di un insieme $A \subset \mathbb{R}$ 
  se: 

  $$
    \forall \mathcal{U}(x)~~~\dot{\mathcal{U}}(x) \cap A \neq \emptyset
  $$
\end{defineimp}

\begin{center}
  \begin{tikzpicture}[scale=1.2]

  \draw[thick] (-3,0) -- (3,0);
  \draw[thick] (-3,0.15) -- (-3,-0.15);
  \draw[thick,->] (2.9,0) -- (3.3,0);

  \node[below] at (1,-0.25) {$x_0$};

  \foreach \x in {-5, -2.3, -1.4, -0.8, -0.4, -0.15, 0.05, 0.18, 0.28, 0.34}
  {
    \filldraw[cmapteal-300] ({1 + \x/3},0) circle (2.2pt);
  }

  \filldraw[cmapteal-500] (1,0) circle (2pt);

  \end{tikzpicture}
\end{center}

\begin{thm}
  $x \in \overline{\mathbb{R}}$ è un punto di accumulazione di $A$ $\iff$ $\exists\{a_n\}$, $a_n \neq x~\forall n \in \mathbb{R}~\colon~a_n \to x$.
\end{thm}

\begin{proof}\leavevmode
  \begin{itemize}
    \item ($\Leftarrow$, $x \in \mathbb{R}$): per ipotesi, si ha che $a_n \to x$. Quindi, per definizione di limite, 
          $$
            \forall \mathcal{U}(x)~\exists n_0 \in \mathbb{N}~\colon~a_n \in \mathcal{U}(x)~\forall n\geqslant n_0
          $$
          quindi:
          $$
            \forall \mathcal{U}(x)~~~\dot{\mathcal{U}}(x) \cap A \neq \emptyset
          $$
    \item ($\Rightarrow$, $x \in \mathbb{R}$): per ipotesi, si ha che, fissato $\mathcal{U}(x)$:
          $$
            \dot{\mathcal{U}}(x) \cap A \neq \emptyset
          $$
          Sia:
          $$
            \mathcal{U}(x) = \left(x - \dfrac{1}{n}, x + \dfrac{1}{n}\right)
          $$
          allora:
          $$
            \exists a_n \in A~\colon~a_n \neq x~\wedge a_n \in \mathcal{U}(x)
          $$
          $$
            0 < |a_n - x| < \dfrac{1}{n}
          $$
          quindi $a_n \to x$ per confronto.
  \end{itemize}
\end{proof}


\begin{cor}
  Se $x$ è un punto di accumulazione di $A$, si ha che $\dot{\mathcal{U}}(x) \cap A$ contiene infiniti 
  elementi.
\end{cor}

\begin{define}[Punto isolato]
  Dato un insieme $A \subset \mathbb{R}$, $x \in A$ è un \textbf{punto isolato} di $A$ se:

  $$ 
    \exists \mathcal{U}(x)~\colon~\mathcal{U}(x) \cap A = \{x\}
  $$
\end{define}

\begin{center}
  \begin{tikzpicture}[scale=1.2]

  \draw[thick] (-3,0) -- (3,0);
  \draw[thick] (-3,0.15) -- (-3,-0.15);
  \draw[thick,->] (2.9,0) -- (3.3,0);

  \node[below] at (1,-0.25) {$x_0$};

  \foreach \x in {-5, -4, -3, -2, 2, 3, 4, 5}
  {
    \filldraw[cmapteal-300] ({1 + \x/3},0) circle (2.2pt);
  }

  \filldraw[cmapteal-500] (1,0) circle (2pt);

  \end{tikzpicture}
\end{center}

\begin{define}[Punto di frontiera]
  Dato $x \in \mathbb{R}$ e $A \subset \mathbb{R}$, $x$ è un \textbf{punto di frontiera} 
  di $A$ se:

  $$
    \forall \mathcal{U}(x)~~~\mathcal{U}(x) \cap A \neq \emptyset ~~~ \wedge ~~~ \mathcal{U}(x) \cap A^C \neq \emptyset
  $$
\end{define}

\begin{center}
  \begin{tikzpicture}[scale=1.2]

  \draw[thick] (-3,0) -- (3,0);
  \draw[thick] (-3,0.15) -- (-3,-0.15);
  \draw[thick,->] (2.9,0) -- (3.3,0);

  \node[below] at (1,-0.25) {$x_0$};

  \foreach \x in {-5, -2.3, -1.4, -0.8, -0.4, -0.15}
  {
    \filldraw[cmapteal-300] ({1 + \x/3},0) circle (1.8pt);
  }

  \foreach \x in {5, 2.3, 1.4, 0.8, 0.4, 0.15}
  {
    \filldraw[tyrian] ({1 + \x/3},0) circle (1.8pt);
  }

  \filldraw[cmapteal-500] (1,0) circle (2.6pt);

  \end{tikzpicture}
\end{center}

\begin{obs}
  I punti isolati sono punti di frontiera banali
\end{obs}

\begin{define}[Punto interno]
  Sia $A \subset \mathbb{R}$, $x \in A$ è un \textbf{punto interno} di $A$ se:

  $$
    \exists \mathcal{U}(x)~\colon~\mathcal{U}(x) \subset A
  $$
\end{define}

\begin{center}
  \begin{tikzpicture}[scale=1.2]

  \draw[ultra thick, cmapteal-300] (-3,0) -- (3,0);
  \draw[thick] (-3,0.15) -- (-3,-0.15);
  \draw[thick,->] (3,0) -- (3.3,0);

  \node[below] at (1,-0.25) {$x_0$};

  \filldraw[cmapteal-500] (1,0) circle (2pt);

  \end{tikzpicture}
\end{center}


\begin{define}[Insieme aperto]
  $A \subset \mathbb{R}$ si definisce \textbf{aperto} se ogni $a \in A$ è punto interno di $A$.
\end{define}

\begin{define}[Insieme chiuso]
  $A \subset \mathbb{R}$ si definisce \textbf{chiuso} contiene tutti i suoi punti di accumulazione finiti.
\end{define}


\begin{es}\leavevmode
  \begin{itemize}
    \item $\{a\}$ è chiuso
    \item $\mathbb{R}$ e $\emptyset$ sono sia aperti che chiusi
  \end{itemize}
\end{es}

\begin{thm}
  Un insieme $A$ è aperto $\iff$ $A^C$ è chiuso.
\end{thm}

\begin{proof}

\end{proof}

\begin{thm}
  Un insieme $A$ è chiuso $\iff$ Se $\forall a_n \in A$ t.che $a_n \to x$ $x \in A$.
\end{thm}

\begin{proof}

\end{proof}


\begin{define}[Compattezza sequenziale]
  Un insieme $A \subset \mathbb{R}$ si definisce \textbf{compatto} se $\forall a_n \in A~\exists x \in A$ ed 
  esiste una sottosuccessione $a_{n_k} \to x$
\end{define}

\begin{thm}[Heine-Borel]
  Un insieme $A \subset$ è compatto $\iff$ $A$ è chiuso e limitato.
\end{thm}

\begin{proof}\leavevmode
  \begin{itemize}
    \item ($\Leftarrow$): Sia $x_n \in A$. Dato che $A$ è limitato, $x_n$ è limitata.
          Quindi per il teorema di Bolzano-Weierstrass, $\exists x_{n_k}~\colon~x_{n_k} \to x \in \mathbb{R}$.
          Quindi $x$ è un punto di accumulazione di $A$. Dato che $A$ è chiuso, contiene tutti i suoi 
          punti di accumulazione, quindi $x \in A$.
    \item ($\Rightarrow$):
  \end{itemize}
\end{proof}


\section{Limiti}

\subsection{Introduzione}

\begin{defineimp}[Limite]
  Sia $f: \mathbb{R} \to \mathbb{R}$ di dominio $\mathfrak{D}$ e sia $x_0 \in \overline{\mathbb{R}}$ 
  un punto di accumulazione di $\mathfrak{D}$, si dice che $f$ ammette \textbf{limite} uguale a $l \in \overline{\mathbb{R}}$ per $x$ 
  che tende a $x_0$ se:

  $$
    \forall \mathcal{U}(l)~\exists \mathcal{U}(x_0)~\colon~\text{ se } x \in \mathfrak{D} \cap \dot{\mathcal{U}}(x_0) \implies f(x) \in \mathcal{U}(l)
  $$
\end{defineimp}

\begin{notation}
  $\displaystyle \lim_{x\to x_0}f(x) = l$, $f(x) \underset{x\to x_0}{\to} l$
\end{notation}

\begin{define}[Limite successionale]
  Data $f: \mathbb{R} \to \mathbb{R}$ di dominio $\mathfrak{D}$, si dice che $f$ ammette \textbf{limite successionale} uguale 
  a $l \in \overline{\mathbb{R}}$ per $x$ che tende a $x_0$ se:

  $$
    \forall a_n \in \mathfrak{D}~\colon~a_n\neq x_0,~a_n \to x_0~~~f(a_n) \to l
  $$
\end{define}

\begin{notation}
  $\displaystyle \slim_{x\to x_0}f(x) = l$
\end{notation}


\begin{thm}[Teorema traghettatore]
  $\displaystyle \lim_{x\to x_0}f(x) = l \iff \slim_{x\to x_0}f(x) = l$
\end{thm}

\begin{proof}\leavevmode
  \begin{itemize}
    \item ($\Rightarrow$): per ipotesi, fissato $\mathcal{U}(l)$:
          $$
            \exists \mathcal{U}(x_0)~\colon~\text{ se } x \in \mathfrak{D} \cap \dot{\mathcal{U}}(x_0) \implies f(x) \in \mathcal{U}(l)
          $$
          sia $a_n \in \mathfrak{D}$ t.che $a_n \neq x_0$, $a_n \to x_0$, si ha che:

          $$
            a_n \in \dot{\mathcal{U}}(x_0)~\definit
          $$

          quindi:

          $$
            f(a_n) \in \mathcal{U}(l)~\definit
          $$
    \item ($\Leftarrow$): per l'implicazione inversa si mostrerà che $\lnot \text{Th.} \implies \lnot \text{Hp.}$ \\
          La negazione della tesi è:

          $$
            \exists \mathcal{U}(l)~\colon~\forall \mathcal{U}(x_0)~\exists x \in \mathfrak{D}\cap \dot{\mathcal{U}}(x_0)~\colon~ f(x) \not \in \mathcal{U}(l)
          $$

          mentre la negazione dell'ipotesi è:

          $$
            \exists a_n \in \mathfrak{D},~a_n \neq x_0,~a_n \to x_0~\colon~f(a_n) \not \to l
          $$

          sia $\mathcal{U}(x_0) = \left(x_0 - \dfrac{1}{n}, x_0 + \dfrac{1}{n}\right)$, fissato $n \in \mathbb{N}~\exists a_n \in \mathfrak{D}\cap \dot{\mathcal{U}}(x_0)\colon f(a_n) \not \in \mathcal{U}(l) \implies f(a_n) \not \to l$. 
          La possibilità di costruire $a_n$ è garantita dall'assioma della scelta.
  \end{itemize}
\end{proof}


\subsection{Teoremi sui limiti}

Molti dei teoremi sui limiti di successioni hanno una "controparte" per i limiti di funzioni 
reali ottenibili tramite il teorema traghettatore.

\begin{thm}[Unicità del limite]
  Sia $f: \mathfrak{D}(f) \to \mathbb{R}$ una funzione reale e sia $x_0$ un punto di accumulazione di $\mathfrak{D}(f)$. Se $\exists l \in \mathbb{R}$ t.che $\lim_{x \to x_0}f(x) = l$, $l$ è unico.
\end{thm}

\begin{proof}
  Dato che $\exists \lim_{x \to x_0}f(x) = l$, allora $\slim_{x\to x_0}f(x) = l$. Quindi:

  $$
    \forall a_n \neq x_0,~a_n \to x_0~~~f(a_n) \to l
  $$

  la tesi segue dal teorema di unicità del limite per le successioni.
\end{proof}

\begin{thm}[Permanenza del segno]
  Sia $f: \mathfrak{D}(f) \to \mathbb{R}$ una funzione reale e sia $x_0$ un punto di accumulazione di $\mathfrak{D}(f)$. Sia inoltre $l := \lim_{x \to x_0}f(x) \in \mathbb{R}$ \\
  Se $f \geqslant 0$ in un $\dot{\mathcal{U}}(x_0)$ $\implies$ $l \geqslant 0$. \\
  Se $l > 0$ $\implies$ $\exists \mathcal{U}(x_0)$ t.che $f(x) > 0~\forall x \in \dot{\mathcal{U}}(x_0) \cap \mathfrak{D}(f)$
\end{thm}

\begin{thm}[Teorema del confronto]
  Siano $f, g, h: \dot{\mathcal{U}}(x_0) \to \mathbb{R}$ t.che $f \leqslant h \leqslant g~~~\forall x \in \dot{\mathcal{U}}(x_0)$, allora, se $\lim_{x \to x_0} f(x) = \lim_{x \to x_0} g(x) = l \in \mathbb{R}$:

  $$
    \lim_{x \to x_0}h(x) = l
  $$
\end{thm}

\begin{thm}[Operazioni con i limiti]
  Siano $f, g$ due funzioni reali e $x_0$ un punto di accumulazione del dominio di entrambe, le 
  stesse proprietà che valgono per i limiti delle successioni valgono per i limiti di $f$ e $g$.
\end{thm}


\begin{obs}
  Siano $f$ e $g$ due funzioni reali, se $x_0$ è punto di accumulazione di $\mathfrak{D}(f)$ e $\mathfrak{D}(g)$, 
  non è detto che sia punto di accumulazione di $\mathfrak{D}(f+g)$.
\end{obs}

\begin{es}
  $f(x) = \sqrt{x}$, $g(x) = \sqrt{-x}$, $x_0 = 0$
\end{es}


\begin{define}[$\diamondsuit$]
  Dato $A \subset \mathbb{R}$, $x \in \mathbb{R}$ è \textbf{punto di accumulazione destro} di $A$ se:

  $$
    \forall \mathcal{U}(x)~\dot{\mathcal{U}}(x) \cap A \cap (x, +\infty) \neq \emptyset
  $$
\end{define}

\begin{define}[$\diamondsuit$]
  Sia $f: \mathbb{R} \to \mathbb{R}$ di dominio $\mathfrak{D}$ e $x_0$ un punto di accumulazione destro di $\mathfrak{D}$, 
  $f$ ha \textbf{limite destro} $l$ per $x$ che tende a $x_0$ se:

  $$
    \forall a_n \in \mathfrak{D}~\colon~a_n > x_0~\wedge~a_n \to x_0 \implies f(a_n) \to l
  $$
\end{define}

\begin{notation}[$\diamondsuit$]
  $\displaystyle \lim_{x\to x_0^+}f(x) = l$
\end{notation}

\begin{define}
  $x_0$ è un \textbf{punto di accumulazione bilatero} di $A$ se è un sia un punto di accumulazione destro che un punto di accumulazione 
  sinistro di $A$.
\end{define}

\begin{obs}
  Se $x_0$ è un punto di accumulazione bilatero:

  $$
    \exists \lim_{x\to x_0}f(x) \iff \lim_{x\to x_0^+}f(x) = \lim_{x\to x_0^-}f(x)
  $$
\end{obs}


\subsection{Continuità}

\begin{defineimp}[Continuità]
  Sia $f: \mathbb{R} \to \mathbb{R}$ una funzione di dominio $\mathfrak{D}$ e sia 
  $x_0 \in \mathfrak{D}$, $f$ è \textbf{continua} in $x_0$ se:

  $$
    \forall \mathcal{U}(f(x_0))~\exists \mathcal{U}(x_0)~\colon~\text{ se } x \in \mathfrak{D} \cap \mathcal{U}(x_0) \implies f(x) \in \mathcal{U}(f(x_0))
  $$
\end{defineimp}

\begin{obs}
  La definizione è equivalente a dire che: \\
  \begin{itemize}
    \item Se $x_0$ è un punto di accumulazione di $\mathfrak{D}$, allora $\displaystyle \lim_{x\to x_0}f(x) = f(x_0)$ 
    \item Se $x_0$ è un punto isoltato di $\mathfrak{D}$, allora $f$ è continua in $x_0$
  \end{itemize}
\end{obs}

\begin{define}[Continuità successionale]
  Sia $f:\mathbb{R} \to \mathbb{R}$ una funzione di dominio $\mathfrak{D}$ e sia $x_0 \in \mathfrak{D}$, 
  $f$ è \textbf{continua per successioni} in $x_0$ se:

  $$
    \forall a_n \in \mathfrak{D},~a_n \to x_0~~~f(a_n) \to f(x_0)
  $$
\end{define}


\begin{thm}
  $f$ è continua in $x_0$ $\iff$ $f$ è continua per successioni in $x_0$
\end{thm}

\begin{proof}\leavevmode
  \begin{itemize}
    \item Se $x_0$ è un punto isolato si prenda $a_n = x_0~\definit$. Allora $f(a_n) \to f(x_0)$
    \item Se $x_0$ è un punto di accumulazione, si ha che $\displaystyle \lim_{x\to x_0}f(x) = f(x_0) \iff \slim_{x\to x_0}f(x) = f(x_0)$
  \end{itemize}
\end{proof}

\begin{define}[$\diamondsuit$]
  Sia $x_0 \in \mathfrak{D}(f)$ t.che $x_0$ è un punto di accumulazione destro di $\mathfrak{D}(f)$, 
  $f$ è \textbf{continua da destra} in $x_0$ se:

  $$
    \lim_{x\to x_0^+}f(x) = f(x_0)
  $$
\end{define}

\begin{prop}
  Se $x_0 \in \mathfrak{D}(f)$ è un punto di accumulazione bilatero, allora $f$ 
  è continua in $x_0$ $\iff$ $f$ è continua sia da destra che da sinistra in $x_0$. 
\end{prop}

\begin{proof}

\end{proof}


\begin{thm}[Permanenza del segno per funzioni continue]
  Sia $f$ continua in $x_0$ e $f(x_0) > 0$, allora:

  $$
    \exists \mathcal{U}(x_0)~\colon~\text{ se } x \in \mathfrak{D}\cap \mathcal{U}(x_0) \implies f(x) > 0
  $$
\end{thm}

\begin{proof}\leavevmode
  \begin{itemize}
    \item Se $x_0$ è un punto isolato, sia $\mathcal{U}(x_0) = \{x_0\}$
    \item Se $x_0$ è un punto di accumulazione, segue dal teorema della permanenza del segno
  \end{itemize}
\end{proof}

\begin{thm}[Continuità per le operazioni]
  Siano $f$ e $g$ continue in $x_0$, allora:

  \begin{enumerate}[I]
    \item $[f + g]$ è continua in $x_0$
    \item $[f\cdot g]$ è continua in $x_0$
    \item se $g(x_0) \neq 0$, $\left[\dfrac{f}{g}\right]$ è continua in $x_0$
  \end{enumerate}
\end{thm}

\begin{proof}
  Sia $a_n \in \mathfrak{D}(f + g)$, con $a_n \to x_0$. Dato che $f$ e $g$ sono continue in $x_0$:

  $$
    f(a_n) \to f(x_0)
  $$
  $$
    g(a_n) \to g(x_0)
  $$

  quindi:

  $$
    [f + g](a_n) = f(a_n) + g(a_n) \to f(x_0) + g(x_0) = [f + g](x_0)
  $$

  In modo analogo si dimostrano i punti II e III.
\end{proof}

\begin{thm}[Continuità della composta]
  Sia $f$ continua in $x_0$ e $g$ continua in $y_0 = f(x_0)$, allora la funzione composta 
  $h = g \circ f$ è continua in $x_0$ 
\end{thm}

\begin{proof}
  Sia $a_n \in \mathfrak{D}(h)$, $a_n \to x_0$ ($a_n \in \mathfrak{D}(f)$, $f(a_n) \in \mathfrak{D}(g)$). \\ 
  Dato che $f$ è continua in $x_0$:

  $$
    f(a_n) \to f(x_0)
  $$

  dato che $g$ è continua in $f(x_0)$

  $$
    g(f(a_n)) \to g(f(x_0))
  $$

  quindi:

  $$
    h(a_n) = g(f(a_n)) \to g(f(x_0)) = h(x_0)
  $$
\end{proof}


\begin{define}
  Una funzione $f: \mathbb{R} \to \mathbb{R}$ è \textbf{globalmente continua} se è continua in ogni $x \in \mathfrak{D}(f)$.
\end{define}


\begin{define}[Punto di discontinuità]
  $x_0 \in \mathfrak{D}(f)$ è un \textbf{punto di discontinuità} di $f$ se $f$ non è continua 
  in $x_0$
\end{define}



\begin{define}[Discontinuità eliminabile (III specie)]
  $x_0$ è un punto di \textbf{discontinuità eliminabile} di $f$ se:

  $$
    \exists \lim_{x\to x_0} f(x) = l \in \mathbb{R}~~~\text{ ma }~~~l \neq f(x_0)
  $$
\end{define}

\begin{es}

\end{es}

\begin{center}
  \begin{tikzpicture}
  \begin{axis}[
    width=12cm,
    height=7cm,
    xmin=-20, xmax=20,
    ymin=-0.3, ymax=1.1,
    axis lines=middle,
    axis line style={->, thick},
    xlabel={$x$},
    ylabel={$y$},
    domain=-20:20,
    samples=400,
    smooth,
    enlargelimits=false,
    ticks=none,
  ]

  \addplot[cmapteal-300, very thick] ({x},{sin(deg(x))/x});

  \addplot[
    only marks,
    mark=o,
    mark size=3pt,
    cmapteal-300,
    line width=1pt
  ] coordinates {(0,1)};
  \addplot[cmapteal-300, only marks, mark=*] coordinates {(0,0)};

  \end{axis}
  \end{tikzpicture}
\end{center}


\begin{define}[Discontinuità salto (I specie)]
  $x_0$ è un punto di \textbf{discontinuità salto} di $f$ se:

  $$
    \exists \lim_{x\to x_0^-} f(x) = l_1 \in \mathbb{R}~~~~\exists \lim_{x\to x_0^+} f(x) = l_2 \in \mathbb{R}~~~\text{ ma }~~~l_1 \neq l_2
  $$
\end{define}

\begin{es}

\end{es}

\begin{center}
  \begin{tikzpicture}
  \begin{axis}[
    width=10cm,
    height=6cm,
    xmin=-3, xmax=3,
    ymin=-1.5, ymax=1.5,
    axis lines=middle,
    axis line style={->, thick},
    xlabel={$x$},
    ylabel={$y$},
    ticks=none
  ]

  \addplot[cmapteal-300, very thick, domain=-3:0] {-1};

  \addplot[cmapteal-300, very thick, domain=0:3] {1};

  \addplot[only marks, mark=o, cmapteal-300, mark size=3pt] coordinates {(0,-1)};

  \addplot[only marks, mark=*, cmapteal-300, mark size=2pt] coordinates {(0,0)};

  \addplot[only marks, mark=o, cmapteal-300, mark size=3pt] coordinates {(0,1)};

  \end{axis}
  \end{tikzpicture}
\end{center}


\begin{define}[Discontinuità di II specie]
  $x_0$ è un punto di \textbf{discontinuità di II specie} di $f$ se:

  $$
    \lim_{x\to x_0^-} f(x) = \pm \infty~(\text{o } \not \exists)~~~~\lim_{x\to x_0^+} f(x) = \pm \infty~(\text{o } \not \exists)
  $$
\end{define}

\begin{es}

\end{es}

\begin{center}
  \begin{tikzpicture}
  \begin{axis}[
    width=12cm,
    height=7cm,
    xmin=-10, xmax=10,
    ymin=-0.3, ymax=1.1,
    axis lines=middle,
    axis line style={->, thick},
    xlabel={$x$},
    ylabel={$y$},
    domain=-10:10,
    samples=400,
    smooth,
    enlargelimits=false,
    ticks=none,
  ]

  \addplot[cmapteal-300, very thick, domain=0:10] ({x},{1/x});
  \addplot[cmapteal-300, very thick, domain=-10:0] ({x}, 0);

  \addplot[cmapteal-300, only marks, mark=*] coordinates {(0,0)};

  \end{axis}
  \end{tikzpicture}
\end{center}

\begin{es}

\end{es}

\begin{center}
  \begin{tikzpicture}
  \begin{axis}[
    width=8cm,
    height=7cm,
    xmin=-2, xmax=2,
    ymin=-2, ymax=2.5,
    axis lines=middle,
    axis line style={->, thick},
    xlabel={$x$},
    ylabel={$y$},
    domain=-2:2,
    samples=400,
    smooth,
    enlargelimits=false,
    ticks=none,
  ]

  \addplot[cmapteal-300, very thick, domain=0.001:2] ({x},{sin(deg(1/x))});
  \addplot[cmapteal-300, very thick, domain=-2:0] ({x}, -x);

  \addplot[
    only marks,
    mark=o,
    mark size=3pt,
    cmapteal-300,
    line width=1pt
  ] coordinates {(0,0)};

  \addplot[cmapteal-300, only marks, mark=*] coordinates {(0,2)};

  \end{axis}
  \end{tikzpicture}
\end{center}


\begin{thm}[$\diamondsuit$]
  Sia $f: (a,b) \to \mathbb{R}$ una funzione monotona crescente e sia $c \in (a,b)$, 
  allora esistono in $c$ entrambi i limiti destro e sinistro e:

  $$
    \lim_{x \to x_0^-}f(x) \leqslant f(x_0) \leqslant \lim_{x \to x_0^+}f(x)
  $$
\end{thm}

\begin{center}
  \begin{tikzpicture}
    \begin{groupplot}[
      group style={
        group size=2 by 2, % 2 columns, 2 rows
        horizontal sep=2cm,
        vertical sep=2cm,
      },
      width=7cm, height=6cm,
      axis lines=middle,
      axis line style={->, thick},
      xlabel={$x$}, ylabel={$y$},
      ticks=none,
      enlargelimits=false,
      samples=400,
      % smooth,
    ]

      % --- Subgraph 1: Top Left ---
      \nextgroupplot[title={I}, xmin=-2, xmax=2, ymin=-2, ymax=2.5]
      \addplot[blue, thick, domain=-2:0] {(1/2*x+1)^2 - 1.5};
      \addplot[blue, thick, domain=0:2] {-1/4*(x-2)^2 + 1.5};
      \addplot[
        only marks,
        mark=o,
        mark size=3pt,
        blue,
        line width=1pt
      ] coordinates {(0, -0.5)};
      \addplot[
        only marks,
        mark=o,
        mark size=3pt,
        blue,
        line width=1pt
      ] coordinates {(0, 0.5)};
      \addplot[only marks, mark=*, blue] coordinates {(0,0)};
      \node[below left] at (0,0) {$x_0$};

      % --- Subgraph 2: Top Right ---
      \nextgroupplot[title={II}, xmin=-2, xmax=2, ymin=-2, ymax=2.5]
      \addplot[teal, thick, domain=-2:0] {sqrt(abs(1/2*x+1))-2};
      \addplot[teal, thick, domain=0:2] {-sqrt(abs(1/2*x-1))+2};
      \addplot[
        only marks,
        mark=o,
        mark size=3pt,
        teal,
        line width=1pt
      ] coordinates {(0, -1)};
      \addplot[
        only marks,
        mark=o,
        mark size=3pt,
        teal,
        line width=1pt
      ] coordinates {(0, 1)};
      \addplot[only marks, mark=*, teal] coordinates {(0,0)};
      \node[below left] at (0,0) {$x_0$};

      % --- Subgraph 3: Bottom Left ---
      \nextgroupplot[title={III}, xmin=-2, xmax=2, ymin=-2, ymax=2.5]
      \addplot[amber, thick, domain=-2:0] {sqrt(abs(1/2*x+1))-1.5};
      \addplot[amber, thick, domain=0:2] {-1/4*(x-2)^2 + 1.5};
      \addplot[
        only marks,
        mark=o,
        mark size=3pt,
        amber,
        line width=1pt
      ] coordinates {(0, -0.5)};
      \addplot[
        only marks,
        mark=o,
        mark size=3pt,
        amber,
        line width=1pt
      ] coordinates {(0, 0.5)};
      \addplot[only marks, mark=*, amber] coordinates {(0,0)};
      \node[below left] at (0,0) {$x_0$};

      % --- Subgraph 4: Bottom Right ---
      \nextgroupplot[title={IV}, xmin=-2, xmax=2, ymin=-2, ymax=2.5]
      \addplot[tyrian, thick, domain=0:2] {sqrt(x)};
      \addplot[tyrian, thick, domain=-2:0] {exp(x) - 1};
      \addplot[only marks, mark=*, tyrian] coordinates {(0,0)};
      \node[below left] at (0,0) {$x_0$};

    \end{groupplot}
  \end{tikzpicture}
\end{center}


\subsection{Prolungamento per continuità}

Sia $f: \mathfrak{D}(f) \to \mathbb{R}$ e sia $x_0 \not \in \mathfrak{D}(f)$ un punto di accumulazione di $\mathfrak{D}(f)$. Non essendo definita in $x_0$, 
non ha senso porsi domande sulla continuità di $f$ in $x_0$. È tuttavia lecito domandarsi se sia possibile definire una funzione $\hat{f}: \mathfrak{D}(f) \cup \{x_0\} \to \mathbb{R}$ t.che $\hat{f}(x) = f(x)~\forall x \neq x_0$ e t.che 
$\hat{f}$ è continua in $x_0$. Tale funzione, se esiste, si dice \textbf{estensione per continuità} di $f$ in $x_0$. \\
Perché $\hat{f}$ sia continua in $x_0$, si deve avere:

$$
  \lim_{x \to x_0^-}\hat{f}(x) = \lim_{x \to x_0^+}\hat{f}(x) = \hat{f}(x_0)
$$

dunque, dato che $\hat{f}$ è uguale a $f$ in un qualsiasi intorno bucato di $x_0$:

$$
  \lim_{x \to x_0^-}f(x) = \lim_{x \to x_0^+}f(x) = \hat{f}(x_0) \in \mathbb{R}
$$

quindi, detto $l := \lim_{x \to x_0}f(x)$, è sufficiente definire $\hat{f}$ come segue:

$$
  \hat{f}(x) =  \begin{cases*}
                  f(x)~~~\text{ se } x \neq x_0 \\
                  l~~~\text{ se } x = x_0
                \end{cases*}
$$

Si noti che è possibile definire $\hat{f}$ solo se $l$ esiste ed è finito.


\section{Teoremi sulle funzioni continue}

\begin{thmimp}[Teorema degli zeri]
  Sia $f: [a,b] \to \mathbb{R}$ una funzione continua. Se $f(a)\cdot f(b) < 0$, allora 

  $$
    \exists c \in [a, b]~\colon~f(c) = 0
  $$
\end{thmimp}

\begin{proof}
  Si supponga che sia $f(a) > 0$ e $f(b) < 0$. Sia $P := \{x \in [a,b]~\colon~f(x) > 0\}$. Dato che $P$ è limitato e non vuoto ($a$ appartiene a $P$), ammette estremo superiore. 
  Sia $c := \sup_P$, esiste $x_{n} \in P$ t.che $x_n \to c$. Si ha inoltre che $f(x_n) > 0~\forall n \in \mathbb{N}$. Dato che $f$ è continua:

  $$
    f(x_n) \to f(c) \geqslant 0~(\text{per il th. della perm. del segno})
  $$

  Sia ora $x'_n := c + \dfrac{1}{n}$. Si ha che $f(x'_n) < 0$ e $x'_n \to c$, quindi:

  $$
    f(x'_n) \to f(c) \leqslant 0~(\text{per il th. della perm. del segno})
  $$

  quindi si conclude che $f(c) = 0$.
\end{proof}

\begin{center}
  \begin{tikzpicture}[
    dot/.style={fill, circle, inner sep=1.5pt},
    >=Stealth
  ]

  % --- 1. Define Coordinates and Draw Axes ---
  \def\xmin{0}
  \def\xmax{6}
  \def\ymin{-3}
  \def\ymax{4}
  \def\a{1}
  \def\b{5}
  \def\fa{3}
  \def\fb{-2}
  \def\c{3.2} % The root c where f(c)=0

  % Draw the axes
  \draw[->] (\xmin, 0) -- (\xmax, 0) node[right] {$x$};
  \draw[->] (0, \ymin) -- (0, \ymax) node[above] {$y$};
  \node at (0, 0) [below left] {$0$};

  % --- 2. Draw the Continuous Function f(x) ---
  \draw[thick, cmapteal-300, smooth]
      (\a, \fa) -- (\c, 0) -- (\b, \fb);

  % --- 3. Mark and Label the Key Points ---

  % Mark a and f(a)
  \draw[dashed, gray] (\a, 0) -- (\a, \fa);
  \draw[dashed, gray] (0, \fa) -- (\a, \fa);
  \node[dot, label={below:$a$}] at (\a, 0) {};
  \node[dot, label={left:$f(a)$}] at (0, \fa) {};
  \node[dot, cmapteal-300, label={above right:$(\text{a}, \text{f}(\text{a}))$}] at (\a, \fa) {};

  % Mark b and f(b)
  \draw[dashed, gray] (\b, 0) -- (\b, \fb);
  \draw[dashed, gray] (0, \fb) -- (\b, \fb);
  \node[dot, label={below:$b$}] at (\b, 0) {};
  \node[dot, label={left:$f(b)$}] at (0, \fb) {};
  \node[dot, cmapteal-300, label={below right:$(\text{b}, \text{f}(\text{b}))$}] at (\b, \fb) {};

  % Mark c (the root)
  \node[dot, amber, label={below:$c$}] at (\c, 0) {};
  \node[dot, amber, label={above right:$(\text{c}, 0)$}] at (\c, 0) {};

  \end{tikzpicture}
\end{center}


\begin{obs}
  Sia $A \subset \mathbb{R}$ t.che $\forall x,y \in A,~x < y \implies [x, y] \subset A$ $\iff$ $A$ è un intervallo
\end{obs}


\begin{define}[Proprietà di Darboux]
  Sia $\mathfrak{I}$ un intervallo non banale, $f:\mathfrak{I} \to \mathbb{R}$ ha 
  la \textbf{proprietà di Darboux} (o è di Darboux, o ha la proprietà dei valori intermedi) se:

  $$
    \forall x,y \in \mathfrak{I},~x < y~~~f \text{ assume tutti i valori tra } f(x) \text{ e } f(y) \text{ nell'intervallo } [x,y] 
  $$
\end{define}

\begin{obs}
  Se $f: \mathfrak{I} \to \mathbb{R}$ è di Darboux:

  $$
    \forall \mathfrak{J} \subset \mathfrak{I} \text{ si ha che } f(\mathfrak{J}) \text{ è un intervallo }
  $$
\end{obs}

\begin{obs}
  Se $f: \mathfrak{I} \to \mathbb{R}$ è di Darboux, non può avere discontinuità eliminabili.
\end{obs}

\begin{thmimp}[Teorema di Darboux]
  Sia $f:\mathfrak{I} \to \mathbb{R}$, se $f$ è continua allora è di Darboux.
\end{thmimp}

\begin{proof}
  Si vuole mostrare che, fissati $x, y \in \mathfrak{I},~x < y$:

  $$
    \forall \gamma~\colon~f(x) < \gamma < f(y)~\exists c \in [x, y]~\colon~f(c) = \gamma
  $$

  \begin{center}
    \begin{tikzpicture}
      \begin{axis}[
          domain=0:5,
          ymin=-1.5,
          ymax=1.5,
          xlabel=$x$,
          smooth,thick,
          axis lines=center,
          xtick=\empty,
          ytick=\empty
        ]

        \addplot[color=cmapteal-300, very thick]{4*e^(-x)*sin(deg(x))*ln(x)};

        \draw[->,color=amber] (0, {4*e^(-1.5)*sin(deg(1.5))*ln(1.5)}) -- (5, {4*e^(-1.5)*sin(deg(1.5))*ln(1.5)});
        \node[left,color=amber] at (4.5,{4*e^(-1.5)*sin(deg(1.5))*ln(1.5)+0.15}) {$\gamma$};

        \coordinate (X) at (1, 0);
        \coordinate (Y) at (2.5, {4*e^(-2.5)*sin(deg(2.5))*ln(2.5)});
        % \coordinate (Z) at (1.5, {1.5*ln(1.5)});

        \draw[dashed, gray] (1, 0) -- (X);
        \draw[dashed, gray] (2.5, 0) -- (Y);

        % Mark points
        \addplot[only marks, mark=*] coordinates {(1, 0)};
        \addplot[only marks, mark=*] coordinates {(2.5, {4*e^(-2.5)*sin(deg(2.5))*ln(2.5)})};
        \addplot[only marks, mark=*] coordinates {(1.5, {4*e^(-1.5)*sin(deg(1.5))*ln(1.5)})};

        % Labels
        \node[below] at (axis cs:1,-0.025) {$x$};
        \node[below] at (axis cs:2.5,0) {$y$};
        \node[below] at (axis cs:1.5,0) {$t$};

        % \node[above] at ($(0.0, -0.6) + (X)$) {$f(x_0)$};
        % \node[above] at ($(0.0, -0.6) + (Y)$) {$f(t)$};
        % \node[above] at ($(0.0, -0.6) + (Z)$) {$f(x_0 + h)$};

      \end{axis}
    \end{tikzpicture}
  \end{center}

  \begin{itemize}
    \item Se $f(x) = f(y)$, la tesi è verificata
    \item Se $f(x) \neq f(y)$, si supponga $f(x) < f(y)$. Si fissi $\gamma$ t.che $f(x) < \gamma < f(y)$. 
          Sia $g(t) = f(t) - \gamma$. Si ha che $g$ è continua (somma tra una funzione continua e una costante). Inoltre:

          $$g(x) = f(x) - \gamma < 0$$
          $$g(y) = f(y) - \gamma > 0$$

          quindi per il teorema degli zeri si ha che:

          $$
            \exists c \in [x, y]~\colon~g(c) = 0 \implies f(c) - \gamma = 0 \implies f(c) = \gamma
          $$
  \end{itemize}

\end{proof}


\begin{lemma}
  Presa $x_n \in [a, b]$, $\exists x \in [a, b]$ ed una sottosuccessione $x_{n_k}$ t.che $x_{n_k} \to x$.
\end{lemma}

\begin{proof}
  Sia $x_n \in [a, b]$, dato che $[a, b]$ è limitato, $x_n$ è limitata. Quindi per il teorema di Bolzano-Weierstrass, 
  $\exists x_{n_k}~\colon~x_{n_k} \to x \in \mathbb{R}$. Dato che $x_n \in [a, b]$, per il teorema della permanenza del segno si ha che:
  
  $$x_{n_k} \geqslant a \implies x \geqslant a$$
  $$x_{n_k} \leqslant b \implies x \leqslant b$$

  quindi $x \in [a,b]$.
\end{proof}


\begin{thmimp}[Teorema di Weierstrass]
  Sia $f:[a, b] \to \mathbb{R}$ una funzione continua, $f$ ammette massimo e minimo.
\end{thmimp}

\begin{proof}
  Sia $M := \sup_f$ (o $+\infty$ se f è illimitata). Si ha che:

  $$
    \exists x_n \in [a,b]~\colon~f(x_n) \to M
  $$
  Per il lemma precedentemente dimostrato, $\exists x \in [a, b]$ e $\exists x_{n_k} \to x \in [a, b]$

  $$
    f(x_n) \to M \implies f(x_{n_k}) \to M
  $$

  ma $x_{n_k} \to x$ e, dato che $f$ è continua:

  $$
    f(x_{n_k}) \to f(x) = M = \max{f}
  $$

  si può procedere in modo analogo per il minimo.
\end{proof}


\begin{cor}
  $f:[a,b] \to \mathbb{R}$ continua $\implies$ $f([a, b]) = [m, M]~~~m = \min_f,~M = \max_f$ 
\end{cor}

\begin{proof}
  Per il teorema di Weierstrass, $m, M \in f([a, b])$. Per il teorema di Darboux, $f([a, b]) = [m, M]$.
\end{proof}


\begin{thm}
  Sia $f:[a, b] \to \mathbb{R}$ monotona, allora: \\

  $f$ è continua $\iff$ $f([a, b]) = [m, M]~~~m = \min_n,~M = \max_f$
\end{thm}

\begin{proof}
  ($\Leftarrow$): Se $f$ non fosse continua, $f$ avrebbe un salto (poiché monotona), quindi $[m, M]$ avrebbe un "buco".
\end{proof}

\begin{cor}
  Sia $f:[a,b] \to [c,d]$ strettamente monotona, allora $f$ è invertibile e $f^{-1}:[c,d] \to [a,b]$ è continua.
\end{cor}

\begin{proof}
  La dimostrazione segue dal fatto che $f^{-1}$ è monotona.
\end{proof}


\begin{prop}
  Sia $f:[a,b] \to [m,M]$ continua e invertibile $\implies$ $f$ è strettamente monotona.
\end{prop}


\section{Continuità uniforme}


\begin{define}[Continuità uniforme]
  Sia $f: \mathfrak{I} \to \mathbb{R}$ una funzione reale, $f$ è \textbf{uniformemente continua} se:

  $$
    \forall \varepsilon > 0~\exists \delta > 0~\colon~\text{ se }~|x - y| < \delta \implies |f(x) - f(y)| < \varepsilon~\forall x,y \in \mathfrak{I}
  $$
\end{define}

\begin{obs}
  Tale definizione non è equivalente a richiedere che, fissato $x \in \mathfrak{I}$:

  $$
    \forall \varepsilon > 0~\exists \delta > 0~\colon~\text{ se }~|x - y| < \delta \implies |f(x) - f(y)| < \varepsilon~\forall y \in \mathfrak{I}
  $$

  in tal caso infatti, la scelta di $\delta = \delta(x)$ può dipendere da $x$.
\end{obs}

\begin{es}
  $f: (0, 1] \to \mathbb{R}$, $f(x) = \dfrac{1}{x}$ non è uniformemente continua.
\end{es}


\begin{thm}[Teorema di Heine-Cantor]
  Sia $f: \mathfrak{I} \to \mathbb{R}$, se $f$ è continua $\implies$ $f$ è uniformemente continua.
\end{thm}

\begin{proof}

\end{proof}


\section{Continuità di Lipschitz}

\begin{define}[Funzione Lipschitz continua]
  Sia $f: [a, b] \to \mathbb{R}$ una funzione reale, $f$ si dice \textbf{Lipschitz continua} se:

  $$
    \exists L \in \mathbb{R}^+~\colon~\forall x, y \in [a, b]~~~|f(x) - f(y)| \leqslant L|x - y|
  $$
\end{define}